%% ===================================================================
\section{Conclusion: where the snakes lie}\label{sec:conclusion}
%% ===================================================================

\subsection{The fundamental obstacle}

The bipartite graph $G_n = (V, H, E)$ has a deceptively simple structure:
edges are given by divisibility, degrees are $M/k + O(1)$, and the
overall expansion ratio $|H|/|V| = M/N \to \infty$.  Yet proving
Hall's condition $|\NH(S)| \geq |S|$ for \emph{all} $S \subseteq V$
has resisted 43~analytic approaches.

The root cause is a single structural fact:

\begin{quote}
\emph{For vertices~$k$ near $N \approx n/2$, the degree
$d(k) \approx \de \approx 2$--$3$ is comparable to the maximum
codegree $D_2 \approx \de$.  Thus $D/D_2 \not\to \infty$.}
\end{quote}

This means:
\begin{itemize}
\item \textbf{Probabilistic methods fail} because they require
  $D / D_2 \to \infty$ (LLL, nibble, Janson all need degree to dominate
  codegree).
\item \textbf{Cauchy--Schwarz fails} because $E_2 \approx E_1^2 / |S|$
  (the codegree sum is comparable to the ``budget,'' leaving no margin).
\item \textbf{Spectral methods fail} because the graph is extremely
  irregular (degree varies from~$\de$ to~$M$), destroying spectral
  concentration.
\item \textbf{Sieve methods fail} because divisibility events are
  positively correlated, and the correlation structure is too complex
  for Bonferroni truncation.
\end{itemize}

\subsection{Why the gap is at exactly $2/\sqrt{e}$}

The current state of knowledge is:
\[
  \left(\frac{2}{\sqrt{e}} + o(1)\right) n\sqrt{\frac{\ln n}{\ln\ln n}}
  \;\leq\; f(n)
  \;\leq\; (1.7398\cdots + o(1))\, n\sqrt{\ln n}.
\]
The gap between lower and upper bounds is a factor of~$\sqrt{\ln\ln n}$.
The constant $C = 2/\sqrt{e}$ in the lower bound arises from a precise
balance.  The Erd\H{o}s--Pomerance argument shows that with
$L = (C - \eps) n \sqrt{\ln n / \ln\ln n}$, there are too few multiples
for the smooth numbers near~$N$ to be matched.  The parameter $\de = 2M/n - 1$ equals
$(2C - 2 + o(1))\sqrt{\ln n / \ln\ln n}$, and at $C = 2/\sqrt{e}$ this
gives $\de \sim 2(2/\sqrt{e} - 1)\sqrt{\ln n / \ln\ln n} \approx
0.426\sqrt{\ln n / \ln\ln n}$.

The growth rate $\sqrt{\ln n / \ln\ln n}$ is \emph{sublogarithmic}:
slower than any power of $\log n$, let alone polynomial.  This means
$\de(n)$ passes through the critical integer values $2, 3, 4, \ldots$
at enormous values of~$n$:

\begin{center}
\begin{tabular}{@{}cc@{}}
\toprule
$\de \geq$ & approximate $n$ \\
\midrule
2 & $3{,}600$ \\
3 & $2 \times 10^7$ \\
4 & $5 \times 10^{14}$ \\
5 & $\sim 10^{26}$ \\
\bottomrule
\end{tabular}
\end{center}

Any proof that requires ``$\de$ large'' (say $\de \geq 10$) would need
$n \gg 10^{100}$, far beyond computational verification.  The gap
between what can be verified ($n \leq 10^6$, where $\de < 3$) and what
can be proved asymptotically (requiring $\de \to \infty$) is a
\emph{desert} of moderate~$n$ where neither tool reaches.

\subsection{What a successful proof would need}

Based on our investigation, a proof of the upper bound~\eqref{eq:upper-bound}
would likely need one of:

\begin{enumerate}
\item \textbf{A new Hall certificate.}
  Some combinatorial or algebraic structure in $G_n$ that certifies
  $|\NH(S)| \geq |S|$ for all~$S$ simultaneously, without checking
  subsets individually.  The FMC approach comes closest but requires
  $\alpha(V_{\mathrm{rest}}) \geq 2$, which remains unproved.

\item \textbf{A new counting technique.}
  Something beyond Cauchy--Schwarz that controls the codegree sum
  $E_2$ more tightly.  The weighted CS (Section~\ref{subsec:weighted-cs})
  shows that the ``right'' weights exist and give margin~$\sim\!1.3$,
  but no analytic expression for them is known.

\item \textbf{A GCD graph argument.}
  The Koukoulopoulos--Maynard framework
  (Section~\ref{subsec:mult-energy}) is the most promising modern tool.
  The adversarial set is multiplicatively unstructured at large~$n$,
  which is the regime where the K--M expansion argument should work.
  Making this quantitative for the specific Erd\H{o}s~710 graph is the
  main open challenge.

\item \textbf{A topological or algebraic argument.}
  Hall's theorem is equivalent to the non-vanishing of a certain
  permanent.  Techniques from algebraic combinatorics (e.g., the
  Combinatorial Nullstellensatz) might certify this without subset
  enumeration.
\end{enumerate}

\subsection{Open questions}

\begin{enumerate}
\item Is $\alpha(V_{\mathrm{rest}}) \geq 2$ for all $n$ sufficiently large?
  (This would close the gap via the FMC theorem.)

\item Does the optimal weighted CS ratio $d^\top C^{-1} d / |S|$ remain
  bounded away from~$1$ as $n \to \infty$?  (It is $\sim\!1.3$ at all
  tested~$n$.)

\item Can the Koukoulopoulos--Maynard GCD graph technique be made
  quantitative enough to prove Hall for $G_n$?

\item Is there a ``forbidden subgraph'' characterization of the tight
  sets of~$G_n$?

\item Can the Shearer entropy bound be made analytic (i.e., can $H(\tau)$
  be bounded from below for all subsets)?
\end{enumerate}

\subsection{Summary table of approaches}

Table~\ref{tab:approaches} lists all 43~approaches investigated,
organized by category, with a one-line failure reason for each.

\begin{longtable}{@{}rlp{5.5cm}@{}}
\caption{Summary of 43 approaches to proving global Hall's condition.}
\label{tab:approaches} \\
\toprule
\# & Approach & Failure reason \\
\midrule
\endfirsthead
\toprule
\# & Approach & Failure reason \\
\midrule
\endhead
\midrule
\multicolumn{3}{r}{\footnotesize\emph{continued on next page}} \\
\endfoot
\bottomrule
\endlastfoot
\multicolumn{3}{l}{\textbf{Cauchy--Schwarz family}} \\
1 & Standard CS & Ratio 0.86 for adversarial $S$ \\
2 & Per-interval CS & Works locally; 87--93\% overlap kills global \\
3 & Weighted CS ($f = 1/\tau$) & Passes computationally; no analytic expression \\
4 & Optimal weighted CS ($C^{-1}d$) & Passes computationally; $C^{-1}$ intractable \\
5 & Filtered CS & Ratio 0.991 (threshold removes key targets) \\
6 & Truncated CS (Ford cap) & Ratio 0.41--0.49 \\
7 & Variable-constant CS & $C_{\mathrm{crit}} = 1.01$ stable, not $\to 1$ \\
\addlinespace
\multicolumn{3}{l}{\textbf{Matching \& fractional methods}} \\
8 & Greedy matching & 112/220 failures; locally optimal $\neq$ globally \\
9 & Uniform fractional & $\min w(k) = 0.06$; starves min-degree vertices \\
10 & Nonuniform fractional (LP) & Exists but doesn't constitute proof for all $n$ \\
11 & FMC with dyadic intervals & $\Sigma > 1$ for $n < 15$K (greedy) \\
12 & FMC three-block & Blocked on $\alpha(S_+)$ proof \\
13 & FMC $V_{\min}/V_{\mathrm{rest}}/S_-$ & Blocked on $\alpha(V_{\mathrm{rest}}) \geq 2$ \\
14 & Sinkhorn iteration & 13--15\% fractional deficiency \\
15 & CLP factoring & Graph not regular; truncation doesn't help \\
16 & $\sqrt{2}$-partition sequential & Local condition passes; global guarantee fails \\
\addlinespace
\multicolumn{3}{l}{\textbf{Probabilistic methods}} \\
17 & Symmetric LLL & $P \cdot e \cdot (D+1) \approx 10^4$ \\
18 & Target-centered LLL & Same; high-$\tau$ targets cause failure \\
19 & Janson inequality & $\Delta \gg (\sum P_i)^2$ \\
20 & Erd\H{o}s--Spencer weighted LLL & Circular obstruction in $x_i$ system \\
21 & Semi-random nibble (Z115) & 25--30\% of $|V|$ permanently unmatchable \\
\addlinespace
\multicolumn{3}{l}{\textbf{Sieve \& inclusion-exclusion}} \\
22 & Bonferroni (order 2) & Waste ratio 2.4--9.2$\times$, grows with $n$ \\
23 & Bonferroni (higher orders) & Signs diverge; no useful truncation \\
24 & Product-formula sieve & Overestimates $|\NH(S)|$ (wrong direction) \\
25 & Unique multiple sieve & 30--90\% of elements have $d_1 = 0$ \\
26 & Standard sieve (Selberg/Brun) & 3.56$\times$ safety factor, wrong direction \\
\addlinespace
\multicolumn{3}{l}{\textbf{Graph-theoretic \& spectral}} \\
27 & Spectral gap & Graph too irregular; $\sigma_2/\sigma_1$ large \\
28 & Haxell independent transversal & Overlap 87--93\% makes conflict graph dense \\
29 & Tur\'an bound & 27--32\% of target; gap grows with $n$ \\
30 & Degeneracy bound & 10--28\% of target; degeneracy $\sim 15\sqrt{n}$ \\
31 & Ford divisor cap & Ratio 0.41--0.49 \\
32 & Multiplicative energy (K--M) & Qualitative but not quantitative for this graph \\
\addlinespace
\multicolumn{3}{l}{\textbf{Partition \& structural}} \\
33 & Dyadic partition ($\times 2$) & 1--2 intervals fail degree condition \\
34 & $\sqrt{2}$-partition & All pass but no global guarantee \\
35 & Fine partition ($\times 1.1$) & All pass locally; same overlap problem \\
36 & Stratified Hall (octile) & 7/8 pass; Q0 fails until subdivided \\
37 & $V_{\min}$ disjointness & Works for $V_{\min}$; doesn't extend to $V_{\mathrm{rest}}$ \\
\addlinespace
\multicolumn{3}{l}{\textbf{Other}} \\
38 & Derandomization & No efficient random process to derandomize \\
39 & Modular remainder & Works for specific elements; doesn't generalize \\
40 & Recursive doubling & Targets already claimed \\
41 & Surplus-excess proof & Bonferroni waste kills it \\
42 & Neumann series bound & Codegree matrix not contractive \\
43 & Quasi-independence & Positive correlations in divisibility \\
\end{longtable}

\begin{figure}[ht]
\centering
\includegraphics[width=\textwidth]{fig_approach_summary.pdf}
\caption{Visual summary of all 43 approaches, organized by category.
  Red: dead (fundamental obstruction).  Orange: partial success
  (works locally or computationally).  Green: promising direction
  (not yet quantitative).}
\label{fig:summary}
\end{figure}

\subsection{Final words}

Erd\H{o}s Problem~\#710 has resisted a determined assault with every
tool in the modern combinatorialist's arsenal.  The computational
evidence is overwhelming: Hall's condition holds with zero failures
through $n = 10^6$, and no adversarial subset has ever been found with
expansion ratio below~$1$.  The gap between what we can compute and
what we can prove is a testament to the depth of the problem.

We hope that this detailed investigation---documenting not just what works
but, more importantly, what does \emph{not} work and \emph{why}---will
save future researchers from repeating these dead ends and guide them
toward the techniques most likely to succeed.
