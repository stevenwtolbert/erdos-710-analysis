%% ===================================================================
\section{Matching and fractional methods}\label{sec:matching}
%% ===================================================================

This section documents eight approaches based on constructing matchings
(exact or fractional) in the bipartite graph~$G_n$.  The common theme:
matchings exist (HK confirms this), but no constructive or
fractional argument can \emph{prove} existence for all~$n$.

\subsection{Greedy matching}\label{subsec:greedy}

\paragraph{Method.}
Match elements in order of increasing degree (hardest first).  Each
element~$k$ is assigned to its available multiple with smallest
$\tau_S(m)$ (least contested).

\paragraph{Results.}
Greedy matching fails in 112 out of 220 tested configurations:

\begin{center}
\begin{tabular}{@{}rccc@{}}
\toprule
$n$ & tests & failures & failure rate \\
\midrule
500 & 55 & 20 & 36\% \\
1{,}000 & 55 & 24 & 44\% \\
2{,}000 & 55 & 33 & 60\% \\
3{,}000 & 55 & 35 & 64\% \\
\bottomrule
\end{tabular}
\end{center}

\paragraph{Why it fails.}
Elements with degree~$\sim\!3$--$8$ (e.g., $k = 98 = 2 \cdot 7^2$,
$k = 81 = 3^4$) get stuck because \emph{all} their multiples were
claimed by previously matched degree-$2$ elements.  The greedy makes
locally optimal but globally suboptimal choices.  The failure rate
\emph{grows} with~$n$.

\paragraph{Verdict: DEAD.}

\subsection{Uniform fractional matching}\label{subsec:uniform-frac}

\paragraph{Method.}
Assign weight $x_{k,h} = 1/\tau_S(h)$ to each edge.  The right-side
constraint $\sum_k x_{k,h} = 1$ is automatic.  If the left-side weight
$w(k) = \sum_{h:\, k \mid h} 1/\tau_S(h) \geq 1$ for all~$k$, then
Birkhoff--von~Neumann integrality gives a perfect matching.

\paragraph{Results.}
The condition $w(k) \geq 1$ for all~$k$ is \textbf{false}.  The
minimum $w(k)$ ranges from $0.057$ to $0.087$---far below~$1$.
The worst element~$k^*$ is always a highly composite number near~$N$
with $\deg(k^*) \approx \de \approx 2$--$3$ and all multiples having
$\tau_S(h) \gg 20$.  The uniform weight $1/\tau_S(h)$ starves
large elements: $w(k^*) \approx \deg / \bar{\tau} \approx 2/30
\approx 0.07$.

\paragraph{Verdict: DEAD.}

\subsection{Nonuniform fractional matching}\label{subsec:nonuniform-frac}

\paragraph{Method.}
Solve the LP: find $x_{k,h} \geq 0$ with $\sum_h x_{k,h} \geq 1$
for all~$k$ and $\sum_k x_{k,h} \leq 1$ for all~$h$.  This always
has an optimal solution (since a perfect matching exists), but the
fractional load $\max_h \sum_k x_{k,h}$ may hover near~$1.0$.

\paragraph{Results.}
LP-optimal fractional matchings exist with $\max$ load $= 1.0$
(tight).  However, the dual LP reveals that the bottleneck elements
are exactly the min-degree vertices near~$N$, and the fractional
solution routes flow through heavily shared targets---reflecting the
same structural difficulty.

\paragraph{Verdict: DEAD} as a proof technique (existence of LP
optimum does not constitute a Hall proof for all~$n$).

\subsection{FMC with dyadic intervals}\label{subsec:fmc-dyadic}

\paragraph{Method.}
Partition $V$ into $J \approx \frac{1}{2}\log_2 n$ dyadic intervals
and apply the FMC theorem (Theorem~\ref{thm:fmc}).

\paragraph{Results (Z99).}
The FMC sum $\Sigma = \sum_j 1/\alpha_j$ is:

\begin{center}
\begin{tabular}{@{}rcc@{}}
\toprule
$n$ & $\Sigma$ (greedy) & Status \\
\midrule
500 & 1.109 & FAIL \\
1{,}000 & 1.154 & FAIL \\
5{,}000 & 1.071 & FAIL \\
10{,}000 & 1.016 & FAIL \\
20{,}000 & 0.917 & OK \\
50{,}000 & 0.755 & OK \\
\bottomrule
\end{tabular}
\end{center}

The FMC sum (using greedy $\alpha_j$) crosses below~$1$ at
$n \approx 15{,}000$--$20{,}000$.  The CS-based overestimate
$\sum 1/\CS_j$ is consistently larger than the greedy-based sum
$\sum 1/\alpha_j$, illustrating that CS underestimates
the true expansion within intervals.

\paragraph{Why it fails for small $n$.}
The bottom interval (vertices near~$N$) has $\alpha_j \approx d_{\min}
\approx 3$, contributing $\sim\!0.33$ to the FMC sum.  With
$J \approx 7$--$10$ intervals, even small contributions from other
intervals push the total above~$1$.

\paragraph{Verdict: DEAD} for $n \lesssim 15{,}000$; works computationally
but not provably for larger~$n$.

\subsection{Three-block FMC}\label{subsec:three-block}

\paragraph{Method (Z103).}
Partition $V = R \cup S_+ \cup S_-$ where $R$ = rough numbers ($P(k) > B$),
$S_+ = \{k > B : P(k) \leq B\}$ (smooth, large), and $S_- = \{1,\ldots,B\}$
(smooth, small).  The three-block FMC needs
$1/\alpha(R) + 1/\alpha(S_+) + 1/\alpha(S_-) < 1$.

\paragraph{Results (Z103b).}

\begin{center}
\begin{tabular}{@{}rccccl@{}}
\toprule
$n$ & $\alpha(R)$ & $\alpha(S_+)$ & $\alpha(S_-)$ & FMC & Status \\
\midrule
1{,}000 & 2.00 & 1.31 & 27.3 & 1.302 & FAIL \\
5{,}000 & 2.40 & 1.59 & 75.4 & 1.059 & FAIL \\
10{,}000 & 2.50 & 2.30 & 117.9 & 0.843 & OK \\
50{,}000 & 3.00 & 3.00 & 350.9 & 0.670 & OK \\
\bottomrule
\end{tabular}
\end{center}

The three-block approach improves over dyadic (crossover at $n \approx
10{,}000$ vs $20{,}000$), but $\alpha(S_+)$ remains unproved analytically.
The greedy heuristic overestimates $\alpha$ by 10--15\%, so the true crossover
may be later.

\paragraph{Verdict: DEAD} without an analytic bound on $\alpha(S_+)$.

\subsection{Sinkhorn iteration}\label{subsec:sinkhorn}

\paragraph{Method.}
Apply Sinkhorn's algorithm (alternating row/column normalization) to the
biadjacency matrix to find a doubly stochastic scaling.  If the scaling
converges, Birkhoff's theorem guarantees a perfect matching.

\paragraph{Results (Z113c).}
Sinkhorn converges, but the fractional deficiency (deviation from doubly
stochastic) is 13--15\%.  The minimum row sum after $1{,}000$ iterations
is $0.85$--$0.87$, not reaching~$1$.  The bottleneck is again the
min-degree vertices near~$N$.

\paragraph{Verdict: DEAD.}

\subsection{CLP factoring}\label{subsec:clp}

\paragraph{Method (Z113c).}
Factor the bipartite graph into perfect matchings via the constructive
Lov\'asz--Plummer algorithm.  If $G_n$ has a decomposition into~$\de$
edge-disjoint perfect matchings, the first one suffices.

\paragraph{Results.}
$G_n$ is not regular ($\deg(k)$ varies from~$\de$ to~$M$), so direct
factoring fails.  Even after truncating to the $\de$-regular subgraph
(keeping only $\de$ edges per vertex), the resulting graph is not
bipartite-regular and does not decompose cleanly.

\paragraph{Verdict: DEAD.}

\subsection{$\sqrt{2}$-partition sequential matching}\label{subsec:sqrt2}

\paragraph{Method (Z31g).}
Partition $V$ using ratio $\sqrt{2}$ (intervals $[c^j, c^{j+1})$ with
$c = \sqrt{2}$) instead of ratio~$2$.  Within each finer interval, the
local degree condition $\min_{k \in I_j} \deg(k) \geq
\max_{h \in \NH(I_j)} |\{k \in I_j : k \mid h\}|$ holds, enabling
sequential matching interval by interval.

\paragraph{Results.}
At every tested~$n$ up to $200{,}000$, the $\sqrt{2}$-partition satisfies
the local degree condition in every interval (worst ratio $= 1.00$).
The number of intervals grows from~$7$ to~$14$.

\paragraph{Why it fails as a proof.}
Sequential matching across intervals requires that targets used by
earlier intervals remain available for later intervals.  The 87--93\%
overlap means most targets are shared across intervals, and there is no
guarantee that a greedy interval-by-interval approach leaves enough
targets for subsequent intervals.  This is the same globalization
barrier as in Section~\ref{subsec:per-interval-cs-detail}.

\paragraph{Verdict: DEAD} for the global argument; the local condition
passes everywhere.
