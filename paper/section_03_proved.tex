%% ===================================================================
\section{What is proved}\label{sec:proved}
%% ===================================================================

We have four classes of rigorous results, none of which individually
suffices to prove the upper bound~\eqref{eq:upper-bound} for all~$n$.

\subsection{Exhaustive Hopcroft--Karp verification}\label{subsec:hk}

\begin{theorem}[Computational verification]\label{thm:computational}
For every integer $n \in [4, 10^6]$, the bipartite graph $G_n$
admits a left-saturating matching.  Equivalently, Hall's condition
$|\NH(S)| \geq |S|$ holds for all $S \subseteq V$.
\end{theorem}

\begin{proof}[Method]
At each integer~$n$, we construct $G_n$ explicitly and run the
Hopcroft--Karp algorithm~\cite{HopcroftKarp1973}, which computes a
maximum matching in $O(\sqrt{|V|}\,|E|)$ time. By K\"onig's theorem,
the maximum matching size equals~$|V|$ if and only if Hall's condition
holds for all subsets of~$V$.

The computation was performed in two phases:
\begin{enumerate}
\item \emph{Python implementation} (\texttt{hpc\_z68\_exhaustive\_hk.py}):
  verified $n \in [4, 10{,}000]$ in 71~seconds, with 9{,}985~passes,
  12~skips (trivial cases with $|V| = 0$), and zero failures.
\item \emph{C/OpenMP implementation} (\texttt{hpc\_z68\_hk.c}):
  extended verification to $n = 10^6$, using parallelism across
  multiple cores.  The full run completed with zero failures.
\end{enumerate}

All matching sizes equal~$|V|$ at every tested~$n$.  There is no
marginal case: at every~$n$, the matching saturates completely.
\end{proof}

\begin{remark}
The computation does not merely check a \emph{random sample} of~$n$
values.  It checks \emph{every} integer in $[4, 10^6]$, providing a
certificate that the upper bound holds unconditionally for $n \leq 10^6$.
This is the strongest result we have.
\end{remark}

\subsection{Per-interval Cauchy--Schwarz}\label{subsec:per-interval-cs}

Decompose $V$ into dyadic intervals $I_j = \{k \in V : 2^{j-1} < k \leq 2^j\}$
for $j = 1, \ldots, J$ (with the first and last intervals possibly truncated).

\begin{theorem}[Per-interval Hall]\label{thm:per-interval}
For each dyadic interval~$I_j$ and all $n$ sufficiently large,
\[
  |\NH(S \cap I_j)| \;\geq\; |S \cap I_j|
  \qquad \text{for all } S \subseteq V.
\]
\end{theorem}

The proof proceeds through the chain Z23--Z28 of our investigation:

\begin{enumerate}
\item \textbf{Homogeneity within intervals.}
  For $k, k' \in I_j$, the degrees satisfy $\deg(k)/\deg(k') \in [1, 2]$.
  The average degree within~$I_j$ is $\bar{d}_j \approx M / (3 \cdot 2^{j-1}/2)$.

\item \textbf{Truncated GCD sum.}  Define
  $G_{\mathrm{trunc}}(I_j) = \sum_{\substack{k,k' \in I_j,\, k \neq k' \\
    \lcm(k,k') \leq n+L}} \frac{\gcd(k,k')}{kk'}$.
  This sum controls the codegree contribution to~$E_2$.

\item \textbf{$G_{\mathrm{trunc}} \to 0$.}
  For smooth $k, k' \in [X, 2X)$ with $\lcm(k,k') \leq n+L$, writing
  $k = da$, $k' = db$ with $\gcd(a,b) = 1$, we have $d \geq d^* = X^2/(n+L)
  \to \infty$.  The Hildebrand--Tenenbaum estimate~\cite{HildebrandTenenbaum1993}
  for smooth number counts shows the outer sum
  $\sum_{d \geq d^*} 1/d$ converges to~$0$, giving
  $G_{\mathrm{trunc}} = O(\log\log n / \log n) \to 0$.

\item \textbf{CS ratio $\to \infty$.}
  The effective codegree parameter is
  $C_{\mathrm{eff}} = 1 + 2G_{\mathrm{trunc}} / H_j + \text{correction}$,
  where $H_j = \sum_{k \in I_j} 1/k$ is the harmonic sum.
  Since $G_{\mathrm{trunc}} \to 0$ and $H_j$ is bounded below, we get
  $C_{\mathrm{eff}} \to 1$, whence the CS ratio
  $\CS(I_j) = \bar{d}_j / C_{\mathrm{eff}} \to \infty$.
\end{enumerate}

\begin{remark}
The per-interval result is genuinely strong: within each interval, the
Cauchy--Schwarz bound proves $|\NH(S \cap I_j)| \geq C_j |S \cap I_j|$
with $C_j \to \infty$.  The fundamental difficulty is that this does
\emph{not} imply global Hall's condition, because the neighborhoods
$\NH(S \cap I_j)$ overlap across different intervals (see
Section~\ref{sec:gap}).
\end{remark}

\subsection{$V_{\min}$ pairwise disjointness}\label{subsec:vmin}

Define $V_{\min} = \{k \in V : k > B,\; \deg(k) = d_{\min}\}$, where
$d_{\min} = \floor{2M/n}$ is the minimum degree among elements of~$V$.

\begin{theorem}\label{thm:vmin-disjoint}
For $n$ sufficiently large, all elements of~$V_{\min}$ have pairwise
disjoint neighborhoods: $\NH(\{k_1\}) \cap \NH(\{k_2\}) = \emptyset$
for all distinct $k_1, k_2 \in V_{\min}$.  Consequently,
$\alpha(V_{\min}) = d_{\min}$, where $\alpha(V_{\min}) =
\min_{\emptyset \neq T \subseteq V_{\min}} |\NH(T)| / |T|$.
\end{theorem}

The proof splits into three cases based on the ``smoothness'' of the
elements (whether $P(k) \leq B$ or $P(k) > B$):

\begin{enumerate}
\item \textbf{Smooth $\times$ smooth.}
  For $B$-smooth elements $k_1, k_2$ near~$N$ with
  $\gcd(k_1, k_2) = d$ and coprime quotients $a = k_1/d$, $b = k_2/d$:
  we need $\lcm(k_1, k_2) = dab \leq n+L$ for a shared target to exist.
  Since $k_1, k_2 \approx N$ and both are smooth, the ``coprime pair
  impossibility theorem'' shows this forces $d \geq N^2/(n+L) \to \infty$,
  but then $a, b$ are bounded and the constraint is too restrictive for
  $\delta > 2.2$ (verified for $n \geq 15{,}000$).

\item \textbf{Rough $\times$ rough.}
  For elements with $P(k_i) > B$: if $k_1 = p_1 m_1$ and $k_2 = p_2 m_2$
  with large primes $p_i > B$, then
  $\lcm(k_1, k_2) \geq p_1 p_2 \cdot \lcm(m_1, m_2) / \gcd(p_1 m_1, p_2 m_2)$.
  Since $p_1, p_2 > B \approx \sqrt{n}$ and $k_i \leq N \approx n/2$,
  we get $\lcm \geq N \cdot B \gg n + L$, so $\deg(k_1, k_2) = 0$.

\item \textbf{Smooth $\times$ rough.}
  For $k_s$ ($B$-smooth, near~$N$) and $k_r$ ($P(k_r) > B$, near~$N$):
  $\gcd(k_s, k_r) \leq N/B$ since $P(k_r) > B$ does not divide~$k_s$.
  Then $\lcm(k_s, k_r) \geq N^2 / (N/B) = NB \gg n + L$.
  Verified computationally at $n = 15$K, $20$K, $50$K.
\end{enumerate}

\subsection{The FMC theorem}\label{subsec:fmc}

The \emph{Fractional Matching Condition} (FMC) provides a sufficient
condition for Hall's theorem via a partition argument.

\begin{theorem}[FMC]\label{thm:fmc}
Let $V = V_1 \cup V_2 \cup \cdots \cup V_r$ be a partition of~$V$
into blocks, and let $\alpha_j = \min_{\emptyset \neq T \subseteq V_j}
|\NH(T)| / |T|$ be the expansion ratio of block~$j$.  If
\begin{equation}\label{eq:fmc-condition}
  \sum_{j=1}^{r} \frac{1}{\alpha_j} \;\leq\; 1,
\end{equation}
then Hall's condition $|\NH(S)| \geq |S|$ holds for all $S \subseteq V$.
\end{theorem}

\begin{proof}
For any $S \subseteq V$, write $S_j = S \cap V_j$ and $s_j = |S_j|$.
For each target $h \in \NH(S)$, define
$f(h) = \sum_{j:\, h \in \NH(S_j)} 1/\alpha_j$.
Then:
\begin{enumerate}[label=(\alph*)]
\item $f(h) \leq \sum_{j=1}^r 1/\alpha_j \leq 1$ for every~$h$
  (each indicator is $0$ or~$1$, and the FMC
  condition~\eqref{eq:fmc-condition} applies).
\item $\sum_{h \in \NH(S)} f(h) = \sum_j (1/\alpha_j)|\NH(S_j)|
  \geq \sum_j s_j = |S|$
  (since $|\NH(S_j)| \geq \alpha_j s_j$ by definition of~$\alpha_j$).
\end{enumerate}
Combining: $|S| \leq \sum_h f(h) \leq 1 \cdot |\NH(S)|$, so
$|\NH(S)| \geq |S|$.
\end{proof}

\begin{remark}
The FMC theorem was verified in Z62 and applied extensively in Z99--Z111.
With the $V_{\min} / V_{\mathrm{rest}} / S_-$ partition:
$\alpha(V_{\min}) = d_{\min}$, $\alpha(S_-) \to \infty$, so the FMC sum
becomes $1/d_{\min} + 1/\alpha(V_{\mathrm{rest}}) + o(1)$.
For $n \geq 100{,}000$, the greedy heuristic gives
$\alpha(V_{\mathrm{rest}}) \geq 4$, yielding
$\text{FMC sum} \leq 1/3 + 1/4 + o(1) = 7/12 < 1$.

\emph{However}, the greedy heuristic only provides a \emph{lower bound}
on $\alpha(V_{\mathrm{rest}})$---it does not constitute a proof.
This is the critical gap: no analytic argument proves
$\alpha(V_{\mathrm{rest}}) \geq 2$ for $n \to \infty$.
\end{remark}

\subsection{Small-$s$ regime}\label{subsec:small-s}

\begin{theorem}[Small-$s$ Hall]\label{thm:small-s}
For any fixed $s_0 \geq 1$, there exists $n_0 = n_0(s_0, \eps)$ such
that for all $n \geq n_0$ and all $S \subseteq V$ with $|S| = s \leq s_0$
and $\min(S) > M/(s_0 + 1)$: $|\NH(S)| \geq s$.
\end{theorem}

\begin{proof}
By Proposition~\ref{prop:multiplier-range}, $\tau_S(h) \leq
|\{j \in \mathbb{Z} : 5 \leq j \leq 2(s_0+1)\}| = 2s_0 - 2 =: C_0$
for all $h \in H$.
Using $E_2 \leq C_0 \cdot E_1$:
\[
  |\NH(S)| \;\geq\; \frac{E_1}{C_0}
  \;\geq\; \frac{s \cdot \de}{C_0}.
\]
Since $\de \to \infty$ and $C_0$ is a constant, we get
$|\NH(S)| \geq s$ for $\de \geq C_0$, which holds for all
$n$ sufficiently large.
\end{proof}

More precisely, Regime~1 of the proof covers all $s \leq s_1(n) =
\floor{\de/2}$, which tends to infinity (albeit slowly: $s_1 \sim
(C + \eps - 1)\sqrt{\ln n / \ln\ln n} / 2$).

\subsection{Summary of proved results}

\begin{center}
\begin{tabular}{@{}lll@{}}
\toprule
Result & Range & Method \\
\midrule
$|\NH(S)| \geq |S|$ for all $S \subseteq V$ &
  $n \in [4, 10^6]$ & Exhaustive HK \\
$|\NH(S \cap I_j)| \geq |S \cap I_j|$ &
  $n \to \infty$, each $I_j$ & Per-interval CS \\
$\alpha(V_{\min}) = d_{\min}$ &
  $n \geq 15{,}000$ & Pairwise disjointness \\
$\sum 1/\alpha_j \leq 1 \Rightarrow$ Hall &
  General & FMC theorem \\
$|\NH(S)| \geq |S|$ for $|S| \leq \floor{\de/2}$ &
  $n$ large & Small-$s$ CS \\
\bottomrule
\end{tabular}
\end{center}

None of these results, alone or in combination, proves global Hall's
condition for all~$n$.  The gap is analyzed in Section~\ref{sec:gap}.
