%% ===================================================================
\section{Probabilistic methods}\label{sec:probabilistic}
%% ===================================================================

Probabilistic existence arguments are among the most powerful tools
in combinatorics.  We tried five probabilistic approaches; all fail
because the bipartite graph~$G_n$ has $D \approx D_2$ (degree~$\approx$
maximum codegree for the hard vertices), precisely the regime where
these methods lose power.

\subsection{Symmetric Lov\'asz Local Lemma}\label{subsec:symmetric-lll}

\paragraph{Method.}
Random matching: each $k \in V$ picks $\varphi(k)$ uniformly from its
$\deg(k)$ multiples in~$H$.  The bad event $B_k$ is a collision
($\varphi(k) = \varphi(k')$ for some $k' \neq k$).  The symmetric
LLL~\cite{ErdosSpencer1991} asserts that if $\Pr[B_k] \cdot e \cdot
(d_k + 1) \leq 1$ for all~$k$, where $d_k$ is the number of other
bad events sharing a target with~$B_k$, then a proper injection exists
with positive probability.

\paragraph{Results.}
The symmetric LLL works in only 4 out of 108 tested configurations
(all trivial: top-packed subsets at $s/N = 0.1$).

\begin{center}
\begin{tabular}{@{}rccc@{}}
\toprule
$n$ & $\max \Pr[B_k]$ & $\max d_k$ & $\max$ LLL value \\
\midrule
500 & 1.83 & 224 & 540 \\
1{,}000 & --- & --- & $>1{,}000$ \\
2{,}000 & 2.13 & 868 & 2{,}091 \\
\bottomrule
\end{tabular}
\end{center}

\paragraph{Why it fails.}
Three compounding problems:
(i)~$\Pr[B_k] > 1$ is not meaningful (collision is near-certain under
uniform random matching for hard elements);
(ii)~the dependency degree $d_k \approx s$ for small elements like
$k = 6, 12$ (they share multiples with essentially every other element);
(iii)~the product $\Pr[B_k] \cdot d_k$ \emph{grows} with~$n$.

\paragraph{Verdict: DEAD.}

\subsection{Target-centered LLL}\label{subsec:target-lll}

\paragraph{Method.}
Define bad events on targets instead: $B_h = \{|\{k : \varphi(k) = h\}|
\geq 2\}$ (target~$h$ receives two or more elements).  The dependency
graph is sparser (each $B_h$ depends only on $B_{h'}$ where $h$ and $h'$
share a source).

\paragraph{Results.}
The target-centered formulation has $\Pr[B_h] \leq 1 - (1 - 1/\bar{d})^{\tau(h)}$
$\approx \tau(h)/\bar{d}$, and the dependency degree is $d_h \leq
\sum_{k:\,k \mid h} (\deg(k) - 1)$.  For highly composite targets
($\tau(h) > 50$), we get $\Pr[B_h] \approx 1$, and the LLL condition
$\Pr[B_h] \cdot e \cdot (d_h + 1) \leq 1$ fails catastrophically.

\paragraph{Feasibility check (Z113a).}
$P \cdot e \cdot (D + 1) \approx 10^4$ at $n = 50{,}000$---four orders of
magnitude above the threshold.

\paragraph{Verdict: DEAD.}

\subsection{Janson's inequality}\label{subsec:janson}

\paragraph{Method.}
Janson's inequality~\cite{Janson1998} gives a bound on the probability
that a random subset avoids all ``bad configurations.''  Applied to our
setting with random target assignment, the Janson bound requires
the sum of pairwise dependencies $\Delta = \sum_{(B_i, B_j) \text{ dep.}}
\Pr[B_i \wedge B_j]$ to be small relative to $(\sum \Pr[B_i])^2$.

\paragraph{Results.}
$\Delta \gg (\sum \Pr[B_i])^2$ because the codegree structure is too
dense.  For vertices near~$N$ with $\deg \approx \de \approx 3$,
every pair of targets of~$k$ is shared with other vertices, making
the joint probabilities comparable to the marginal probabilities.

\paragraph{Verdict: DEAD.}

\subsection{Erd\H{o}s--Spencer weighted LLL}\label{subsec:erdos-spencer}

\paragraph{Method.}
The Erd\H{o}s--Spencer~\cite{ErdosSpencer1991} version of the LLL allows
non-uniform probabilities: assign $x_i \in (0, 1)$ to each bad event
$B_i$ such that $\Pr[B_i] \leq x_i \prod_{j \sim i} (1 - x_j)$.

\paragraph{Results.}
The optimal $x_i$ must satisfy a system of $|V|$ nonlinear inequalities.
For our graph, the system is infeasible: the high-codegree vertices
force $x_i \to 1$, which makes $\prod (1 - x_j) \to 0$, creating a
circular obstruction.

\paragraph{Verdict: DEAD.}

\subsection{Semi-random nibble}\label{subsec:nibble}

\paragraph{Method (Z115).}
R\"odl nibble / semi-random process~\cite{Rodl1985,MolloyReed2002}:
each round, every unmatched vertex picks a random available target;
if a target is picked by exactly one vertex, match them.  Repeat
for 500~rounds.  Run HK on the residual graph to check matchability.

\paragraph{Results.}

\begin{center}
\begin{tabular}{@{}rccc@{}}
\toprule
$n$ & avg.\ residual \% & HK on residual & unmatchable / $|V|$ \\
\midrule
1{,}000 & 52.2\% & FAILS & 29.7\% \\
5{,}000 & 43.3\% & FAILS & --- \\
10{,}000 & 38.7\% & FAILS & 29.4\% \\
20{,}000 & 34.4\% & FAILS & --- \\
30{,}000 & 32.3\% & FAILS & 24.8\% \\
50{,}000 & 29.5\% & (not tested) & --- \\
\bottomrule
\end{tabular}
\end{center}

\paragraph{Why it fails.}
The nibble matches $\sim\!65$--$70\%$ of vertices, but the remaining
$30$--$35\%$ include $\sim\!25\%$ of~$|V|$ that are \emph{genuinely
unmatchable} in the residual graph.  HK fails on every residual at every~$n$
and every random seed (variance across seeds is $\pm 1$--$2\%$---the failure
is structural, not bad luck).

\paragraph{Theoretical diagnosis.}
The R\"odl nibble theory requires $D / D_2 \to \infty$ (degree~$\gg$
max codegree).  For our graph, $D \approx D_2 \approx \de$ for the
hard vertices near~$N$.  The theory correctly predicts failure.

When a hard vertex~$k_1$ ($\deg \approx 3$--$5$) randomly claims
target~$h$, other hard vertices~$k_2$ that also needed~$h$ (as one of
their few options) permanently lose a critical target.  After enough
rounds, many vertices have lost \emph{all} viable targets.

\begin{figure}[ht]
\centering
\includegraphics[width=\textwidth]{fig_nibble_residual.pdf}
\caption{Semi-random nibble results.  The residual fraction (blue)
  shrinks with~$n$, but the permanently unmatchable fraction (red)
  remains at $25$--$30\%$ of~$|V|$.  HK fails on every residual
  at every~$n$ tested.}
\label{fig:nibble}
\end{figure}

\paragraph{Verdict: DEAD.}
This is approach \#43 on the dead list.
