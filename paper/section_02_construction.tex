%% ===================================================================
\section{The bipartite graph construction}\label{sec:construction}
%% ===================================================================

\subsection{Parameters}

Fix $\eps > 0$ and define
\begin{align}
  C &= 2/\sqrt{e} \approx 1.2131, \label{eq:C-def} \\
  L &= (C + \eps)\, n\sqrt{\ln n/\ln\ln n}, \label{eq:L-def2} \\
  M &= L - n, \label{eq:M-def} \\
  N &= \lfloor n/2 \rfloor, \label{eq:N-def} \\
  \de(n) &= 2M/n - 1, \label{eq:delta-def} \\
  B &= \lfloor\sqrt{n+L}\,\rfloor. \label{eq:B-def}
\end{align}
Here $L$ is the interval length, $M = |H|$ the number of targets,
$N = |V|$ the pool size, $\de$ the minimum degree parameter,
and $B$ the smoothness bound.
The parameter $\de(n) = 2(C+\eps-1)\sqrt{\ln n/\ln\ln n} + O(1)$
tends to infinity, but slowly: $\de(10{,}000) \approx 2.1$,
$\de(50{,}000) \approx 2.4$, and $\de < 3$ for all $n < 10^7$.

\subsection{The bipartite graph $G_n$}\label{subsec:graph-def}

\begin{definition}\label{def:bipartite-graph}
The \emph{Erd\H{o}s~710 bipartite graph} is $G_n = (V, H, E)$ where:
\begin{itemize}
\item $V = \{1, 2, \ldots, N\}$ (the ``source'' or ``pool'' vertices),
\item $H = (2n,\; n+L] \cap \mathbb{Z}$ (the ``target'' vertices, $|H| = M$),
\item $E = \{(k, h) : k \in V,\; h \in H,\; k \mid h\}$.
\end{itemize}
\end{definition}

This graph arises from the $U/V$ decomposition of $\{1,\ldots,n\}$.

\subsection{The $U/V$ decomposition}

\begin{proposition}[Top half --- doubling map]\label{prop:doubling}
The map $\varphi(k) = 2k$ is a divisible injection from
$U = \{\ceil{n/2}+1,\ldots,n\}$ into $(n, 2n]$.
\end{proposition}

\begin{proof}
For $k \in U$: $k > n/2$ implies $2k > n$; $k \leq n$ implies $2k \leq 2n$;
$k \mid 2k$; distinct~$k$ give distinct~$2k$.
\end{proof}

Since the doubling map uses only targets in $(n, 2n]$, the bottom half~$V$
must be matched into the remaining targets $H = (2n, n+L]$. By Hall's
marriage theorem~\cite{Hall1935}, a perfect matching $V \to H$ exists if and
only if $|\NH(S)| \geq |S|$ for every $S \subseteq V$.

Thus the entire problem reduces to verifying Hall's condition in~$G_n$.

\subsection{Degree structure}\label{subsec:degree}

For $k \in V$ and $h \in H$, the divisibility condition $k \mid h$ means
$h = jk$ for some integer~$j$. Since $h \in (2n, n+L]$, the multiplier
$j = h/k$ satisfies $2n/k < j \leq (n+L)/k$.

\begin{proposition}[Left degrees]\label{prop:left-degree}
For $k \in V$:
\[
  \deg(k) = \floor{\frac{n+L}{k}} - \floor{\frac{2n}{k}}
  = \frac{M}{k} + O(1).
\]
In particular, $\deg(k) \geq \floor{2M/n} \geq \de(n)$ for all $k \leq n/2$.
\end{proposition}

The degree function $d(k) = M/k + O(1)$ is monotonically decreasing in~$k$.
Vertices near $k = 1$ have degree $\sim M$, while vertices near $k = N \approx n/2$
have degree $\sim \de \approx 2$--$3$. This extreme heterogeneity is a
central feature of the problem.

\begin{proposition}[Codegrees]\label{prop:codegree}
For distinct $k_1, k_2 \in V$:
\begin{align*}
  \deg(k_1, k_2) &:= |\{h \in H : k_1 \mid h \text{ and } k_2 \mid h\}| \\
  &= \begin{cases}
    M / \lcm(k_1, k_2) + O(1) & \text{if } \lcm(k_1, k_2) \leq n+L, \\
    0 \text{ or } 1 & \text{if } \lcm(k_1, k_2) > n+L.
  \end{cases}
\end{align*}
\end{proposition}

The codegree equals $M \cdot \gcd(k_1, k_2) / (k_1 k_2) + O(1)$ when the lcm
is small. For pairs of large elements near $N$, we have
$\lcm(k_1, k_2) \approx k_1 k_2 / \gcd(k_1, k_2) \approx n^2 / (4 \gcd)$,
which exceeds $n+L$ unless $\gcd \gg n / \sqrt{\log n}$. Thus \emph{most}
pairs of large elements have codegree~$0$ or~$1$.

\begin{proposition}[Right degrees]\label{prop:right-degree}
For $h \in H$, the ``multiplicity'' or right-degree is
\[
  \tau_V(h) = |\{k \in V : k \mid h\}|.
\]
For $S \subseteq V$, we write $\tau_S(h) = |\{k \in S : k \mid h\}|$.
\end{proposition}

Right degrees vary enormously: highly composite targets (e.g., multiples of
large primorials) have $\tau_V(h) > 100$, while prime or near-prime targets
have $\tau_V(h) = 1$--$3$.

\begin{figure}[ht]
\centering
\includegraphics[width=\textwidth]{fig_degree_distribution.pdf}
\caption{Left degree $d(k)$ vs.\ normalized position $k/N$ for three
  values of~$n$.  The degree drops from $\sim M$ at $k=1$ to
  $\sim\delta \approx 2$--$3$ at $k = N$.}
\label{fig:degree}
\end{figure}

\subsection{Restricted multiplier range}\label{subsec:restricted-mult}

\begin{proposition}\label{prop:multiplier-range}
If $k \in S \subseteq (M/(s+1),\, n/2]$ with $|S| = s$ divides $h \in H$,
then the multiplier $j = h/k$ satisfies $5 \leq j < 2(s+1)$ for $n$
sufficiently large.
\end{proposition}

\begin{proof}
Lower bound: $j = h/k > 2n/(n/2) = 4$, so $j \geq 5$.
Upper bound: $j = h/k < (n+L) / (M/(s+1)) = (n+L)(s+1)/M$.
Since $(n+L)/M = 1 + n/M \to 1$, we get $j < 2(s+1)$ for large~$n$.
\end{proof}

This is a key structural fact: the number of divisors of~$h$ contributing to
$\tau_S(h)$ is at most $|\{j \in \mathbb{Z} : 5 \leq j \leq 2(s+1)\}| = 2s-2$,
which is much smaller than the full divisor function $\tau(h)$.

\subsection{The Cauchy--Schwarz framework}\label{subsec:cs-framework}

The Cauchy--Schwarz inequality gives the fundamental lower bound on
neighborhood size:

\begin{proposition}[CS lower bound]\label{prop:cs-bound}
For any $S \subseteq V$:
\begin{equation}\label{eq:cs-bound}
  |\NH(S)| \;\geq\; \frac{E_1^2}{E_2},
\end{equation}
where $E_1 = \sum_{k \in S} \deg(k) = M\sigma + O(s)$ with
$\sigma = \sum_{k \in S} 1/k$, and
$E_2 = \sum_{h \in \NH(S)} \tau_S(h)^2$.
\end{proposition}

\begin{proof}
By Cauchy--Schwarz: $E_1^2 = \bigl(\sum_h \tau_S(h)\bigr)^2
\leq |\NH(S)| \cdot \sum_h \tau_S(h)^2 = |\NH(S)| \cdot E_2$.
\end{proof}

To prove Hall's condition $|\NH(S)| \geq s$, it suffices to show
$E_1^2 / E_2 \geq s$, or equivalently $E_1^2 \geq s \cdot E_2$.

\subsection{Case A: Small minimum element}

\begin{proposition}[Case~A]\label{prop:case-a}
If $S \subseteq V$ with $|S| = s$ and $\min(S) \leq M/(s+1)$, then
$|\NH(S)| \geq s$.
\end{proposition}

\begin{proof}
Let $a = \min(S)$. Then $\deg(a) \geq \floor{M/a} \geq M/(M/(s+1)) - 1
= s+1-1 = s$.
\end{proof}

Case~A covers all subsets whose smallest element is ``small enough.''
The remaining Case~B, where $\min(S) > M/(s+1)$, is where all the
difficulty resides.
