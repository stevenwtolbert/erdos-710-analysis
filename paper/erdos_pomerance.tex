\documentclass[12pt]{amsart}

%%% Packages
\usepackage{amsmath,amssymb,amsthm}
\usepackage{booktabs}
\usepackage{mathtools}
\usepackage{microtype}
\usepackage{enumitem}
\usepackage{graphicx}
\usepackage{longtable}
\usepackage{array}
\usepackage{float}
\usepackage[colorlinks=true,citecolor=blue,urlcolor=blue,linkcolor=blue]{hyperref}

%%% Fix overflow: allow generous line breaking
\emergencystretch=1.5em
\sloppy

%%% Figures path
\graphicspath{{../../experiments/figures/}}

%%% Theorem environments
\theoremstyle{plain}
\newtheorem{theorem}{Theorem}[section]
\newtheorem{lemma}[theorem]{Lemma}
\newtheorem{proposition}[theorem]{Proposition}
\newtheorem{corollary}[theorem]{Corollary}
\newtheorem{conjecture}[theorem]{Conjecture}
\newtheorem{observation}[theorem]{Observation}

\theoremstyle{definition}
\newtheorem{definition}[theorem]{Definition}

\theoremstyle{remark}
\newtheorem{remark}[theorem]{Remark}
\newtheorem{example}[theorem]{Example}

%%% Math operators and macros
\DeclareMathOperator{\lcm}{lcm}
\DeclareMathOperator{\CS}{CS}
\DeclareMathOperator{\FMC}{FMC}
\newcommand{\NH}{N_H}
\newcommand{\floor}[1]{\left\lfloor #1 \right\rfloor}
\newcommand{\ceil}[1]{\left\lceil #1 \right\rceil}
\newcommand{\abs}[1]{\left| #1 \right|}
\newcommand{\norm}[1]{\left\| #1 \right\|}
\newcommand{\eps}{\varepsilon}
\newcommand{\de}{\delta}

\subjclass[2020]{11B83, 05C70, 11N25, 05D15}

\keywords{Erd\H{o}s Problem 710, divisible injection, Hall's marriage theorem,
  Cauchy--Schwarz, bipartite matching, smooth numbers}

\title[Approaches to Erd\H{o}s--Pomerance]{Computational and analytic approaches to the Erd\H{o}s--Pomerance divisible injection problem}

\author{Steven Tolbert}

\date{February 2026}

\begin{document}

\begin{abstract}
We report on an extensive computational and analytic investigation of
Erd\H{o}s Problem~\#710: determine $f(n)$, the minimum length~$L$ such that
there exists an injection $\varphi\colon \{1,\ldots,n\}\to (n,n+L]\cap\mathbb{Z}$
with $k\mid\varphi(k)$ for all~$k$. The conjectured answer is
$f(n) = (2/\sqrt{e}+o(1))\,n\sqrt{\ln n/\ln\ln n}$, where the lower bound
is due to Erd\H{o}s and Pomerance. We verify computationally that the
matching upper bound holds for all $n\le 10^6$ via exhaustive
Hopcroft--Karp matching with zero failures. For the analytic regime
$n\to\infty$, we document 43~distinct approaches---including Cauchy--Schwarz
variants, fractional matching, the Lov\'asz Local Lemma, spectral methods,
sieve bounds, semi-random nibble, and graph-theoretic techniques---each of
which fails to close the gap between per-interval Hall verification and
global Hall's condition. We identify the precise structural obstruction:
the worst-case subset~$T^*$ spans approximately~48--53\% of vertices,
drawn from \emph{all} dyadic intervals at roughly~60\% density per interval,
with expansion ratio barely exceeding~$1$.
We catalog every approach, its failure mechanism, and the computational
evidence, with the aim of guiding future work on this long-standing open
problem.
\end{abstract}

\maketitle
\tableofcontents

%% ============================================================
%% ===================================================================
\section{Introduction}\label{sec:intro}
%% ===================================================================

\subsection{The problem}

For a positive integer~$n$, define $f(n)$ to be the minimum $L \in \mathbb{Z}_{>0}$
such that there exists an injection
$\varphi\colon \{1,\ldots,n\} \to (n,\, n+L] \cap \mathbb{Z}$
with $k \mid \varphi(k)$ for all~$k$.
Thus $f(n)$ is the length of the shortest interval $(n, n+L]$ admitting
a \emph{divisible injection} from~$\{1,\ldots,n\}$.
This function was studied by Erd\H{o}s and
Pomerance~\cite{ErdosPomerance1980}, and its asymptotic determination
appears as Problem~\#710 on the Erd\H{o}s problems
database~\cite{Erdos1990,Guy2004}.
The requirement $\varphi(k) > n$ ensures that $\varphi(k) \neq k$, and
$k \mid \varphi(k)$ constrains each image to lie among the multiples of~$k$
that exceed~$n$.

The problem sits at the intersection of multiplicative number theory, extremal
combinatorics, and the anatomy of integers. It asks: how densely can we pack
the integers $1,\ldots,n$ into a short interval above~$n$, with each integer
assigned to one of its own multiples?

\subsection{History and prior work}\label{subsec:history}

The problem has its roots in Erd\H{o}s's 1938 Tomsk paper~\cite{Erdos1938}, which
introduced \emph{$2$-primitive sets}---sets where no element divides the product
of two distinct others---and established foundational density estimates.
The Erd\H{o}s--S\'ark\"ozy--Szemer\'edi collaboration~\cite{ErdosSarkozy1966}
produced a series of at least ten papers on divisibility properties of integer
sequences through the 1960s--70s, culminating in precise characterizations
of extremal sets.

The specific function~$f(n)$ and its asymptotic behavior were studied by
Erd\H{o}s and Pomerance~\cite{ErdosPomerance1980}, who established the lower bound
\begin{equation}\label{eq:lower-bound}
  f(n) \;\geq\; \left(\frac{2}{\sqrt{e}} - o(1)\right)
  n\sqrt{\frac{\ln n}{\ln\ln n}}.
\end{equation}
Their argument counts the available multiples of~$k$ in short intervals and shows
that no interval shorter than the right-hand side
of~\eqref{eq:lower-bound} can accommodate a divisible injection.
The constant $2/\sqrt{e} \approx 1.2131$ arises from optimizing over the
``smooth number'' parameter: elements~$k$ whose largest prime factor
$P(k) \leq k^{1/u}$ (with $u = \sqrt{\ln n / \ln\ln n}$) are the bottleneck,
and the Dickman function~$\rho(u)$ governs their density.

Erd\H{o}s and Pomerance also established the best known upper bound:
\begin{equation}\label{eq:known-upper}
  f(n) \;\leq\; (1.7398\cdots + o(1))\, n\sqrt{\ln n}.
\end{equation}
Note the absence of the $\ln\ln n$ correction: the upper
bound~\eqref{eq:known-upper} exceeds the conjectured
answer~\eqref{eq:lower-bound} by a factor of~$\sqrt{\ln\ln n}$,
which diverges (albeit slowly).  Closing this gap---proving the matching
upper bound
\begin{equation}\label{eq:upper-bound}
  f(n) \;\leq\; \left(\frac{2}{\sqrt{e}} + o(1)\right)
  n\sqrt{\frac{\ln n}{\ln\ln n}}
\end{equation}
and thereby determining $f(n)$ asymptotically---has remained open
since~1980. The problem appears in Guy's \emph{Unsolved Problems in
Number Theory}~\cite{Guy2004} (Sections~C9 and~C10) and carries
a \$500 prize on \texttt{erdosproblems.com}.

\subsection{Our contribution}

This paper does not prove~\eqref{eq:upper-bound}. Instead, we report on an
extensive investigation---both computational and analytic---that:
\begin{enumerate}[label=(\alph*)]
\item \textbf{Verifies computationally} that the upper bound holds for
  all $n \leq 10^6$, via exhaustive Hopcroft--Karp maximum bipartite matching
  with zero failures (Section~\ref{sec:proved});
\item \textbf{Proves partial results:} per-interval Hall's condition via
  Cauchy--Schwarz, the $V_{\min}$ disjointness lemma, the FMC theorem, and
  the small-$s$ regime (Section~\ref{sec:proved});
\item \textbf{Documents 43~distinct analytic approaches} to closing the gap
  between per-interval and global Hall's condition, each of which fails for
  identifiable structural reasons
  (Sections~\ref{sec:cs-family}--\ref{sec:graph-theoretic});
\item \textbf{Identifies the precise obstruction:} the worst-case
  subset~$T^*$ spans~$\sim\!48$--$53\%$ of all vertices, drawn from
  every dyadic interval at $\sim\!60\%$ density, with the global
  expansion ratio $\alpha(V) = \min_{S} |\NH(S)|/|S|$ barely
  exceeding~$1$ (Section~\ref{sec:gap});
\item \textbf{Catalogs interesting computational phenomena}---sawtooth
  oscillations, phase transitions, the Shearer--CS dichotomy---that may
  guide future work (Section~\ref{sec:phenomena}).
\end{enumerate}

The spirit of this paper is that of a detailed postmortem
of a prolonged research effort, written so that future researchers need not
repeat the same dead ends. We believe the negative results are themselves
valuable, both for the structural insight they provide and for the sharp
quantitative bounds on where each approach breaks.

\subsection{Why the problem is hard}\label{subsec:why-hard}

The fundamental difficulty can be stated in a single sentence:

\begin{quote}
\emph{The bipartite graph $G_n = (V, H, E)$ has per-interval expansion
$\gg 1$ but global expansion ratio barely exceeding~$1$, and no
known analytic method can certify that the expansion ratio remains
above~$1$ as $n \to \infty$.}
\end{quote}

More precisely:
\begin{itemize}
\item Within each dyadic interval $I_j = \{k \in V : 2^{j-1} < k \leq 2^j\}$,
  the Cauchy--Schwarz bound proves $|\NH(S \cap I_j)| \geq C_j |S \cap I_j|$
  with $C_j \to \infty$ as $n \to \infty$.
\item But the neighborhoods $\NH(S \cap I_j)$ overlap by~87--93\% across
  different intervals~$I_j$, so summing the per-interval guarantees
  \emph{double-counts} the same targets.
\item The global Cauchy--Schwarz ratio $E_1^2 / (|S| \cdot E_2)$ oscillates
  around~$1.0$ for adversarial subsets~$S$, falling as low as $\sim\!0.988$
  at some values of~$n$---below the threshold needed for Hall's condition.
\item The worst-case K\"onig deficient set spans \emph{all} intervals at
  $\sim\!60\%$ density per interval and \emph{all} degree classes,
  defeating any partition-based strategy.
\end{itemize}

The root cause is that the minimum degree~$\delta \approx 2M/n - 1$ is small
(growing only as $\sqrt{\ln n / \ln\ln n}$), and for vertices~$k$ near~$n/2$,
the degree~$d(k) \approx \delta$ is comparable to the maximum codegree~$D_2$.
This means $D/D_2 \not\to \infty$, which is precisely the regime where
probabilistic methods (LLL, nibble, Janson) and algebraic methods (spectral
gap) lose their power.

\subsection{Paper organization}

\begin{itemize}
\item Section~\ref{sec:construction}: The bipartite graph construction and
  its degree structure.
\item Section~\ref{sec:proved}: What is proved---HK verification, per-interval
  CS, $V_{\min}$ disjointness, FMC theorem.
\item Section~\ref{sec:gap}: The critical gap between per-interval and global
  Hall's condition.
\item Section~\ref{sec:cs-family}: The Cauchy--Schwarz family (6+ variants).
\item Section~\ref{sec:matching}: Matching and fractional methods (8+ variants).
\item Section~\ref{sec:probabilistic}: Probabilistic methods (LLL, Janson,
  nibble).
\item Section~\ref{sec:sieve}: Sieve and inclusion-exclusion methods.
\item Section~\ref{sec:graph-theoretic}: Graph-theoretic and spectral methods.
\item Section~\ref{sec:phenomena}: Interesting computational phenomena.
\item Section~\ref{sec:initial}: Results from the initial proof attempt.
\item Section~\ref{sec:conclusion}: Conclusion---where the snakes lie.
\end{itemize}

All computational experiments are reproducible from scripts included in the
\texttt{experiments/} directory of the accompanying code repository. Every
claim in this paper is traced to a specific verification script or state file.

%% ===================================================================
\section{The bipartite graph construction}\label{sec:construction}
%% ===================================================================

\subsection{Parameters}

Fix $\eps > 0$ and define
\begin{align}
  C &= 2/\sqrt{e} \approx 1.2131, \label{eq:C-def} \\
  L &= (C + \eps)\, n\sqrt{\ln n/\ln\ln n}, \label{eq:L-def2} \\
  M &= L - n, \label{eq:M-def} \\
  N &= \lfloor n/2 \rfloor, \label{eq:N-def} \\
  \de(n) &= 2M/n - 1, \label{eq:delta-def} \\
  B &= \lfloor\sqrt{n+L}\,\rfloor. \label{eq:B-def}
\end{align}
Here $L$ is the interval length, $M = |H|$ the number of targets,
$N = |V|$ the pool size, $\de$ the minimum degree parameter,
and $B$ the smoothness bound.
The parameter $\de(n) = 2(C+\eps-1)\sqrt{\ln n/\ln\ln n} + O(1)$
tends to infinity, but slowly: $\de(10{,}000) \approx 2.1$,
$\de(50{,}000) \approx 2.4$, and $\de < 3$ for all $n < 10^7$.

\subsection{The bipartite graph $G_n$}\label{subsec:graph-def}

\begin{definition}\label{def:bipartite-graph}
The \emph{Erd\H{o}s~710 bipartite graph} is $G_n = (V, H, E)$ where:
\begin{itemize}
\item $V = \{1, 2, \ldots, N\}$ (the ``source'' or ``pool'' vertices),
\item $H = (2n,\; n+L] \cap \mathbb{Z}$ (the ``target'' vertices, $|H| = M$),
\item $E = \{(k, h) : k \in V,\; h \in H,\; k \mid h\}$.
\end{itemize}
\end{definition}

This graph arises from the $U/V$ decomposition of $\{1,\ldots,n\}$.

\subsection{The $U/V$ decomposition}

\begin{proposition}[Top half --- doubling map]\label{prop:doubling}
The map $\varphi(k) = 2k$ is a divisible injection from
$U = \{\ceil{n/2}+1,\ldots,n\}$ into $(n, 2n]$.
\end{proposition}

\begin{proof}
For $k \in U$: $k > n/2$ implies $2k > n$; $k \leq n$ implies $2k \leq 2n$;
$k \mid 2k$; distinct~$k$ give distinct~$2k$.
\end{proof}

Since the doubling map uses only targets in $(n, 2n]$, the bottom half~$V$
must be matched into the remaining targets $H = (2n, n+L]$. By Hall's
marriage theorem~\cite{Hall1935}, a perfect matching $V \to H$ exists if and
only if $|\NH(S)| \geq |S|$ for every $S \subseteq V$.

Thus the entire problem reduces to verifying Hall's condition in~$G_n$.

\subsection{Degree structure}\label{subsec:degree}

For $k \in V$ and $h \in H$, the divisibility condition $k \mid h$ means
$h = jk$ for some integer~$j$. Since $h \in (2n, n+L]$, the multiplier
$j = h/k$ satisfies $2n/k < j \leq (n+L)/k$.

\begin{proposition}[Left degrees]\label{prop:left-degree}
For $k \in V$:
\[
  \deg(k) = \floor{\frac{n+L}{k}} - \floor{\frac{2n}{k}}
  = \frac{M}{k} + O(1).
\]
In particular, $\deg(k) \geq \floor{2M/n} \geq \de(n)$ for all $k \leq n/2$.
\end{proposition}

The degree function $d(k) = M/k + O(1)$ is monotonically decreasing in~$k$.
Vertices near $k = 1$ have degree $\sim M$, while vertices near $k = N \approx n/2$
have degree $\sim \de \approx 2$--$3$. This extreme heterogeneity is a
central feature of the problem.

\begin{proposition}[Codegrees]\label{prop:codegree}
For distinct $k_1, k_2 \in V$:
\begin{align*}
  \deg(k_1, k_2) &:= |\{h \in H : k_1 \mid h \text{ and } k_2 \mid h\}| \\
  &= \begin{cases}
    M / \lcm(k_1, k_2) + O(1) & \text{if } \lcm(k_1, k_2) \leq n+L, \\
    0 \text{ or } 1 & \text{if } \lcm(k_1, k_2) > n+L.
  \end{cases}
\end{align*}
\end{proposition}

The codegree equals $M \cdot \gcd(k_1, k_2) / (k_1 k_2) + O(1)$ when the lcm
is small. For pairs of large elements near $N$, we have
$\lcm(k_1, k_2) \approx k_1 k_2 / \gcd(k_1, k_2) \approx n^2 / (4 \gcd)$,
which exceeds $n+L$ unless $\gcd \gg n / \sqrt{\log n}$. Thus \emph{most}
pairs of large elements have codegree~$0$ or~$1$.

\begin{proposition}[Right degrees]\label{prop:right-degree}
For $h \in H$, the ``multiplicity'' or right-degree is
\[
  \tau_V(h) = |\{k \in V : k \mid h\}|.
\]
For $S \subseteq V$, we write $\tau_S(h) = |\{k \in S : k \mid h\}|$.
\end{proposition}

Right degrees vary enormously: highly composite targets (e.g., multiples of
large primorials) have $\tau_V(h) > 100$, while prime or near-prime targets
have $\tau_V(h) = 1$--$3$.

\begin{figure}[ht]
\centering
\includegraphics[width=\textwidth]{fig_degree_distribution.pdf}
\caption{Left degree $d(k)$ vs.\ normalized position $k/N$ for three
  values of~$n$.  The degree drops from $\sim M$ at $k=1$ to
  $\sim\delta \approx 2$--$3$ at $k = N$.}
\label{fig:degree}
\end{figure}

\subsection{Restricted multiplier range}\label{subsec:restricted-mult}

\begin{proposition}\label{prop:multiplier-range}
If $k \in S \subseteq (M/(s+1),\, n/2]$ with $|S| = s$ divides $h \in H$,
then the multiplier $j = h/k$ satisfies $5 \leq j < 2(s+1)$ for $n$
sufficiently large.
\end{proposition}

\begin{proof}
Lower bound: $j = h/k > 2n/(n/2) = 4$, so $j \geq 5$.
Upper bound: $j = h/k < (n+L) / (M/(s+1)) = (n+L)(s+1)/M$.
Since $(n+L)/M = 1 + n/M \to 1$, we get $j < 2(s+1)$ for large~$n$.
\end{proof}

This is a key structural fact: the number of divisors of~$h$ contributing to
$\tau_S(h)$ is at most $|\{j \in \mathbb{Z} : 5 \leq j \leq 2(s+1)\}| = 2s-2$,
which is much smaller than the full divisor function $\tau(h)$.

\subsection{The Cauchy--Schwarz framework}\label{subsec:cs-framework}

The Cauchy--Schwarz inequality gives the fundamental lower bound on
neighborhood size:

\begin{proposition}[CS lower bound]\label{prop:cs-bound}
For any $S \subseteq V$:
\begin{equation}\label{eq:cs-bound}
  |\NH(S)| \;\geq\; \frac{E_1^2}{E_2},
\end{equation}
where $E_1 = \sum_{k \in S} \deg(k) = M\sigma + O(s)$ with
$\sigma = \sum_{k \in S} 1/k$, and
$E_2 = \sum_{h \in \NH(S)} \tau_S(h)^2$.
\end{proposition}

\begin{proof}
By Cauchy--Schwarz: $E_1^2 = \bigl(\sum_h \tau_S(h)\bigr)^2
\leq |\NH(S)| \cdot \sum_h \tau_S(h)^2 = |\NH(S)| \cdot E_2$.
\end{proof}

To prove Hall's condition $|\NH(S)| \geq s$, it suffices to show
$E_1^2 / E_2 \geq s$, or equivalently $E_1^2 \geq s \cdot E_2$.

\subsection{Case A: Small minimum element}

\begin{proposition}[Case~A]\label{prop:case-a}
If $S \subseteq V$ with $|S| = s$ and $\min(S) \leq M/(s+1)$, then
$|\NH(S)| \geq s$.
\end{proposition}

\begin{proof}
Let $a = \min(S)$. Then $\deg(a) \geq \floor{M/a} \geq M/(M/(s+1)) - 1
= s+1-1 = s$.
\end{proof}

Case~A covers all subsets whose smallest element is ``small enough.''
The remaining Case~B, where $\min(S) > M/(s+1)$, is where all the
difficulty resides.

%% ===================================================================
\section{What is proved}\label{sec:proved}
%% ===================================================================

We have four classes of rigorous results, none of which individually
suffices to prove the upper bound~\eqref{eq:upper-bound} for all~$n$.

\subsection{Exhaustive Hopcroft--Karp verification}\label{subsec:hk}

\begin{theorem}[Computational verification]\label{thm:computational}
For every integer $n \in [4, 10^6]$, the bipartite graph $G_n$
admits a left-saturating matching.  Equivalently, Hall's condition
$|\NH(S)| \geq |S|$ holds for all $S \subseteq V$.
\end{theorem}

\begin{proof}[Method]
At each integer~$n$, we construct $G_n$ explicitly and run the
Hopcroft--Karp algorithm~\cite{HopcroftKarp1973}, which computes a
maximum matching in $O(\sqrt{|V|}\,|E|)$ time. By K\"onig's theorem,
the maximum matching size equals~$|V|$ if and only if Hall's condition
holds for all subsets of~$V$.

The computation was performed in two phases:
\begin{enumerate}
\item \emph{Python implementation} (\texttt{hpc\_z68\_exhaustive\_hk.py}):
  verified $n \in [4, 10{,}000]$ in 71~seconds, with 9{,}985~passes,
  12~skips (trivial cases with $|V| = 0$), and zero failures.
\item \emph{C/OpenMP implementation} (\texttt{hpc\_z68\_hk.c}):
  extended verification to $n = 10^6$, using parallelism across
  multiple cores.  The full run completed with zero failures.
\end{enumerate}

All matching sizes equal~$|V|$ at every tested~$n$.  There is no
marginal case: at every~$n$, the matching saturates completely.
\end{proof}

\begin{remark}
The computation does not merely check a \emph{random sample} of~$n$
values.  It checks \emph{every} integer in $[4, 10^6]$, providing a
certificate that the upper bound holds unconditionally for $n \leq 10^6$.
This is the strongest result we have.
\end{remark}

\subsection{Per-interval Cauchy--Schwarz}\label{subsec:per-interval-cs}

Decompose $V$ into dyadic intervals $I_j = \{k \in V : 2^{j-1} < k \leq 2^j\}$
for $j = 1, \ldots, J$ (with the first and last intervals possibly truncated).

\begin{theorem}[Per-interval Hall]\label{thm:per-interval}
For each dyadic interval~$I_j$ and all $n$ sufficiently large,
\[
  |\NH(S \cap I_j)| \;\geq\; |S \cap I_j|
  \qquad \text{for all } S \subseteq V.
\]
\end{theorem}

The proof proceeds through the chain Z23--Z28 of our investigation:

\begin{enumerate}
\item \textbf{Homogeneity within intervals.}
  For $k, k' \in I_j$, the degrees satisfy $\deg(k)/\deg(k') \in [1, 2]$.
  The average degree within~$I_j$ is $\bar{d}_j \approx M / (3 \cdot 2^{j-1}/2)$.

\item \textbf{Truncated GCD sum.}  Define
  $G_{\mathrm{trunc}}(I_j) = \sum_{\substack{k,k' \in I_j,\, k \neq k' \\
    \lcm(k,k') \leq n+L}} \frac{\gcd(k,k')}{kk'}$.
  This sum controls the codegree contribution to~$E_2$.

\item \textbf{$G_{\mathrm{trunc}} \to 0$.}
  For smooth $k, k' \in [X, 2X)$ with $\lcm(k,k') \leq n+L$, writing
  $k = da$, $k' = db$ with $\gcd(a,b) = 1$, we have $d \geq d^* = X^2/(n+L)
  \to \infty$.  The Hildebrand--Tenenbaum estimate~\cite{HildebrandTenenbaum1993}
  for smooth number counts shows the outer sum
  $\sum_{d \geq d^*} 1/d$ converges to~$0$, giving
  $G_{\mathrm{trunc}} = O(\log\log n / \log n) \to 0$.

\item \textbf{CS ratio $\to \infty$.}
  The effective codegree parameter is
  $C_{\mathrm{eff}} = 1 + 2G_{\mathrm{trunc}} / H_j + \text{correction}$,
  where $H_j = \sum_{k \in I_j} 1/k$ is the harmonic sum.
  Since $G_{\mathrm{trunc}} \to 0$ and $H_j$ is bounded below, we get
  $C_{\mathrm{eff}} \to 1$, whence the CS ratio
  $\CS(I_j) = \bar{d}_j / C_{\mathrm{eff}} \to \infty$.
\end{enumerate}

\begin{remark}
The per-interval result is genuinely strong: within each interval, the
Cauchy--Schwarz bound proves $|\NH(S \cap I_j)| \geq C_j |S \cap I_j|$
with $C_j \to \infty$.  The fundamental difficulty is that this does
\emph{not} imply global Hall's condition, because the neighborhoods
$\NH(S \cap I_j)$ overlap across different intervals (see
Section~\ref{sec:gap}).
\end{remark}

\subsection{$V_{\min}$ pairwise disjointness}\label{subsec:vmin}

Define $V_{\min} = \{k \in V : k > B,\; \deg(k) = d_{\min}\}$, where
$d_{\min} = \floor{2M/n}$ is the minimum degree among elements of~$V$.

\begin{theorem}\label{thm:vmin-disjoint}
For $n$ sufficiently large, all elements of~$V_{\min}$ have pairwise
disjoint neighborhoods: $\NH(\{k_1\}) \cap \NH(\{k_2\}) = \emptyset$
for all distinct $k_1, k_2 \in V_{\min}$.  Consequently,
$\alpha(V_{\min}) = d_{\min}$, where $\alpha(V_{\min}) =
\min_{\emptyset \neq T \subseteq V_{\min}} |\NH(T)| / |T|$.
\end{theorem}

The proof splits into three cases based on the ``smoothness'' of the
elements (whether $P(k) \leq B$ or $P(k) > B$):

\begin{enumerate}
\item \textbf{Smooth $\times$ smooth.}
  For $B$-smooth elements $k_1, k_2$ near~$N$ with
  $\gcd(k_1, k_2) = d$ and coprime quotients $a = k_1/d$, $b = k_2/d$:
  we need $\lcm(k_1, k_2) = dab \leq n+L$ for a shared target to exist.
  Since $k_1, k_2 \approx N$ and both are smooth, the ``coprime pair
  impossibility theorem'' shows this forces $d \geq N^2/(n+L) \to \infty$,
  but then $a, b$ are bounded and the constraint is too restrictive for
  $\delta > 2.2$ (verified for $n \geq 15{,}000$).

\item \textbf{Rough $\times$ rough.}
  For elements with $P(k_i) > B$: if $k_1 = p_1 m_1$ and $k_2 = p_2 m_2$
  with large primes $p_i > B$, then
  $\lcm(k_1, k_2) \geq p_1 p_2 \cdot \lcm(m_1, m_2) / \gcd(p_1 m_1, p_2 m_2)$.
  Since $p_1, p_2 > B \approx \sqrt{n}$ and $k_i \leq N \approx n/2$,
  we get $\lcm \geq N \cdot B \gg n + L$, so $\deg(k_1, k_2) = 0$.

\item \textbf{Smooth $\times$ rough.}
  For $k_s$ ($B$-smooth, near~$N$) and $k_r$ ($P(k_r) > B$, near~$N$):
  $\gcd(k_s, k_r) \leq N/B$ since $P(k_r) > B$ does not divide~$k_s$.
  Then $\lcm(k_s, k_r) \geq N^2 / (N/B) = NB \gg n + L$.
  Verified computationally at $n = 15$K, $20$K, $50$K.
\end{enumerate}

\subsection{The FMC theorem}\label{subsec:fmc}

The \emph{Fractional Matching Condition} (FMC) provides a sufficient
condition for Hall's theorem via a partition argument.

\begin{theorem}[FMC]\label{thm:fmc}
Let $V = V_1 \cup V_2 \cup \cdots \cup V_r$ be a partition of~$V$
into blocks, and let $\alpha_j = \min_{\emptyset \neq T \subseteq V_j}
|\NH(T)| / |T|$ be the expansion ratio of block~$j$.  If
\begin{equation}\label{eq:fmc-condition}
  \sum_{j=1}^{r} \frac{1}{\alpha_j} \;\leq\; 1,
\end{equation}
then Hall's condition $|\NH(S)| \geq |S|$ holds for all $S \subseteq V$.
\end{theorem}

\begin{proof}
For any $S \subseteq V$, write $S_j = S \cap V_j$ and $s_j = |S_j|$.
For each target $h \in \NH(S)$, define
$f(h) = \sum_{j:\, h \in \NH(S_j)} 1/\alpha_j$.
Then:
\begin{enumerate}[label=(\alph*)]
\item $f(h) \leq \sum_{j=1}^r 1/\alpha_j \leq 1$ for every~$h$
  (each indicator is $0$ or~$1$, and the FMC
  condition~\eqref{eq:fmc-condition} applies).
\item $\sum_{h \in \NH(S)} f(h) = \sum_j (1/\alpha_j)|\NH(S_j)|
  \geq \sum_j s_j = |S|$
  (since $|\NH(S_j)| \geq \alpha_j s_j$ by definition of~$\alpha_j$).
\end{enumerate}
Combining: $|S| \leq \sum_h f(h) \leq 1 \cdot |\NH(S)|$, so
$|\NH(S)| \geq |S|$.
\end{proof}

\begin{remark}
The FMC theorem was verified in Z62 and applied extensively in Z99--Z111.
With the $V_{\min} / V_{\mathrm{rest}} / S_-$ partition:
$\alpha(V_{\min}) = d_{\min}$, $\alpha(S_-) \to \infty$, so the FMC sum
becomes $1/d_{\min} + 1/\alpha(V_{\mathrm{rest}}) + o(1)$.
For $n \geq 100{,}000$, the greedy heuristic gives
$\alpha(V_{\mathrm{rest}}) \geq 4$, yielding
$\text{FMC sum} \leq 1/3 + 1/4 + o(1) = 7/12 < 1$.

\emph{However}, the greedy heuristic only provides a \emph{lower bound}
on $\alpha(V_{\mathrm{rest}})$---it does not constitute a proof.
This is the critical gap: no analytic argument proves
$\alpha(V_{\mathrm{rest}}) \geq 2$ for $n \to \infty$.
\end{remark}

\subsection{Small-$s$ regime}\label{subsec:small-s}

\begin{theorem}[Small-$s$ Hall]\label{thm:small-s}
For any fixed $s_0 \geq 1$, there exists $n_0 = n_0(s_0, \eps)$ such
that for all $n \geq n_0$ and all $S \subseteq V$ with $|S| = s \leq s_0$
and $\min(S) > M/(s_0 + 1)$: $|\NH(S)| \geq s$.
\end{theorem}

\begin{proof}
By Proposition~\ref{prop:multiplier-range}, $\tau_S(h) \leq
|\{j \in \mathbb{Z} : 5 \leq j \leq 2(s_0+1)\}| = 2s_0 - 2 =: C_0$
for all $h \in H$.
Using $E_2 \leq C_0 \cdot E_1$:
\[
  |\NH(S)| \;\geq\; \frac{E_1}{C_0}
  \;\geq\; \frac{s \cdot \de}{C_0}.
\]
Since $\de \to \infty$ and $C_0$ is a constant, we get
$|\NH(S)| \geq s$ for $\de \geq C_0$, which holds for all
$n$ sufficiently large.
\end{proof}

More precisely, Regime~1 of the proof covers all $s \leq s_1(n) =
\floor{\de/2}$, which tends to infinity (albeit slowly: $s_1 \sim
(C + \eps - 1)\sqrt{\ln n / \ln\ln n} / 2$).

\subsection{Summary of proved results}

\begin{center}
\begin{tabular}{@{}lll@{}}
\toprule
Result & Range & Method \\
\midrule
$|\NH(S)| \geq |S|$ for all $S \subseteq V$ &
  $n \in [4, 10^6]$ & Exhaustive HK \\
$|\NH(S \cap I_j)| \geq |S \cap I_j|$ &
  $n \to \infty$, each $I_j$ & Per-interval CS \\
$\alpha(V_{\min}) = d_{\min}$ &
  $n \geq 15{,}000$ & Pairwise disjointness \\
$\sum 1/\alpha_j \leq 1 \Rightarrow$ Hall &
  General & FMC theorem \\
$|\NH(S)| \geq |S|$ for $|S| \leq \floor{\de/2}$ &
  $n$ large & Small-$s$ CS \\
\bottomrule
\end{tabular}
\end{center}

None of these results, alone or in combination, proves global Hall's
condition for all~$n$.  The gap is analyzed in Section~\ref{sec:gap}.

%% ===================================================================
\section{The critical gap}\label{sec:gap}
%% ===================================================================

The central negative result of our investigation is that no analytic
argument we have found can bridge the gap between per-interval Hall's
condition (proved in Section~\ref{subsec:per-interval-cs}) and global
Hall's condition.  This section characterizes the gap precisely.

\subsection{Per-interval Hall does not imply global Hall}\label{subsec:gap-statement}

Let $I_1, \ldots, I_J$ be the dyadic intervals partitioning~$V$.
Per-interval Hall states that $|\NH(S \cap I_j)| \geq |S \cap I_j|$
for each~$j$.  The naive attempt to globalize is:
\begin{align*}
  |\NH(S)| &= \Bigl|\bigcup_j \NH(S \cap I_j)\Bigr| \\
  &\geq \sum_j |\NH(S \cap I_j)| - \text{overlaps}
  \;\geq\; |S| - \text{overlaps}.
\end{align*}
This works only if the overlap is zero.  In practice, the overlap is enormous.

\subsection{Cross-interval overlap}\label{subsec:overlap}

\begin{observation}[Z29a]\label{obs:overlap}
Adjacent dyadic intervals share $87$--$93\%$ of their targets.
Non-adjacent intervals share similarly.  At $n = 50{,}000$, the total
pairwise overlap exceeds the total surplus by a factor of~$3.3$.
\end{observation}

The data from Z29:

\begin{center}
\begin{tabular}{@{}rrrrrr@{}}
\toprule
$n$ & intervals & $|V|$ & surplus & overlap & margin \\
\midrule
1{,}000 & 4 & 268 & 619 & 998 & $-379$ \\
5{,}000 & 5 & 1{,}373 & 5{,}059 & 11{,}269 & $-6{,}211$ \\
10{,}000 & 6 & 2{,}695 & 12{,}296 & 32{,}243 & $-19{,}947$ \\
50{,}000 & 7 & 13{,}001 & 76{,}395 & 253{,}043 & $-176{,}648$ \\
\bottomrule
\end{tabular}
\end{center}

Here ``surplus'' is $\sum_j (|\NH(I_j)| - |I_j|)$, ``overlap'' is
$\sum_{j < j'} |\NH(I_j) \cap \NH(I_{j'})|$, and ``margin'' is
surplus~$-$~overlap.  The margin is \emph{deeply negative} and worsening
with~$n$.

\subsection{Target multiplicity}\label{subsec:multiplicity}

\begin{observation}[Z29b]\label{obs:multiplicity}
Most targets are shared by multiple intervals.  At $n = 50{,}000$:
$\mu_{\max} = 7$ (equal to the number of intervals), $\mu_{\mathrm{avg}}
= 4.74$, and only $5.3\%$ of targets are unique to one interval.
\end{observation}

A ``max-multiplicity'' bridge argument would require
$\min_j \alpha_j \geq \mu_{\max} \approx \frac{1}{4}\log_2 n$.
Since $\min_j \alpha_j \approx 1.7$--$2.7$ at tested values of~$n$,
this approach fails by an order of magnitude.

\subsection{Global Cauchy--Schwarz failure}\label{subsec:global-cs}

\begin{observation}[Z112g, Z113b, Z114]\label{obs:global-cs}
The global Cauchy--Schwarz ratio $\CS(S) = E_1^2 / (|S| \cdot E_2)$ falls
below~$1$ for adversarial subsets~$S$ at all tested $n \geq 10{,}000$.
\end{observation}

The data from Z114 at the base constant $C = 2/\sqrt{e}$ (with
$\eps = 0.05$, $C_{\mathrm{mult}} = 1.00$):

\begin{center}
\begin{tabular}{@{}rcc@{}}
\toprule
$n$ & $\CS(T_0)$ & $|T_0|/|V|$ \\
\midrule
2{,}000 & 1.027 & 45.4\% \\
5{,}000 & 1.032 & 47.3\% \\
7{,}000 & 1.005 & 48.1\% \\
10{,}000 & \textbf{0.991} & 48.9\% \\
20{,}000 & \textbf{0.992} & 50.3\% \\
50{,}000 & \textbf{0.998} & 52.0\% \\
100{,}000 & \textbf{0.988} & 53.1\% \\
\bottomrule
\end{tabular}
\end{center}

Bold entries are below~$1$: the Cauchy--Schwarz bound \emph{does not}
prove Hall's condition for these subsets.  The values oscillate around~$1.0$
without converging in either direction (see Section~\ref{sec:phenomena}
on the oscillation phenomenon).

\begin{figure}[ht]
\centering
\includegraphics[width=\textwidth]{fig_global_cs_oscillation.pdf}
\caption{Top: global Cauchy--Schwarz ratio $\CS(T_0)$ for the
  adversarial subset at the base constant, oscillating around the
  threshold~$1$.  Bottom: the fraction $|T_0|/|V|$ of vertices in the
  adversarial subset, growing from $45\%$ to $53\%$.}
\label{fig:cs-oscillation}
\end{figure}

\subsection{The K\"onig deficient set}\label{subsec:konig}

The adversarial subset~$T_0$ achieving the minimum CS ratio has a
distinctive structure:

\begin{observation}\label{obs:konig}
The greedy-adversarial subset~$T_0$ spans approximately $48$--$53\%$ of
all vertices in~$V$, drawn from \emph{all} dyadic intervals at roughly
$60\%$ density per interval, and from \emph{all} degree classes.
\end{observation}

This is the core difficulty: the worst-case subset is not localized to
a single interval or degree band---it is a diffuse, structured set that
exploits the entire graph.

\subsection{The expansion ratio}\label{subsec:expansion}

\begin{observation}[Z112n]\label{obs:expansion}
The greedy-adversarial lower bound on the expansion ratio
$\alpha(V) = \min_{\emptyset \neq S \subseteq V} |\NH(S)| / |S|$
lies in the interval $(1.00, 1.25)$ for all tested $n \geq 3{,}000$.
The true minimum may be lower (the greedy heuristic does not
find the worst-case subset), but the exhaustive HK verification
confirms $\alpha(V) \geq 1$ for all $n \leq 10^6$.
\end{observation}

This razor-thin margin---expansion barely exceeding~$1$---is why every
approach fails.  A proof must establish $\alpha(V) \geq 1$ for all~$n$,
but the margin leaves no room for the $O(1)$ errors inherent in
asymptotic arguments.

\subsection{Why the gap persists}\label{subsec:why-gap}

The situation can be summarized as a trilemma:

\begin{enumerate}
\item \textbf{Per-interval analysis is too local.}
  Each interval sees strong expansion ($\CS \to \infty$), but combining
  intervals destroys the guarantee because neighborhoods overlap massively.

\item \textbf{Global analysis is too weak.}
  The global Cauchy--Schwarz bound oscillates around~$1.0$ for adversarial
  subsets, dipping to $\sim\!0.988$ at some values of~$n$---tantalizingly
  close to~$1$ but on the wrong side.

\item \textbf{Partition-based methods need $\alpha(V_{\mathrm{rest}}) \geq 2$.}
  The FMC theorem reduces the problem to bounding the expansion ratio
  of each block.  The $V_{\min}$ block is handled, but no analytic argument
  proves $\alpha(V_{\mathrm{rest}}) \geq 2$, even though the greedy
  heuristic gives $\alpha(V_{\mathrm{rest}}) \geq 2.3$ at all tested~$n$.
\end{enumerate}

The remainder of this paper catalogs 43~approaches that attempt to
resolve this trilemma, organized by technique.

%% ===================================================================
\section{The Cauchy--Schwarz family}\label{sec:cs-family}
%% ===================================================================

The Cauchy--Schwarz inequality is the most natural tool for bounding
neighborhood sizes in bipartite graphs.  We tried six variants,
each addressing a different weakness of the standard bound.  All fail
to prove global Hall, though each illuminates a different aspect of the
problem.

\subsection{Standard Cauchy--Schwarz}\label{subsec:standard-cs}

The standard bound (Proposition~\ref{prop:cs-bound}) gives
$|\NH(S)| \geq E_1^2 / E_2$.  The condition $E_1^2 / (|S| \cdot E_2)
\geq 1$ is equivalent to requiring that the average squared multiplicity
$\overline{\tau^2} := E_2 / |\NH(S)|$ does not exceed $\bar{d}^2 / s$,
where $\bar{d} = E_1 / s$ is the average degree.

\paragraph{Why it fails.}
For adversarial subsets~$S$ in Case~B (with $\min(S) > M/(s+1)$),
the codegree sum $E_2$ includes cross-terms from pairs with small lcm.
The truncated GCD sum $G_{\mathrm{trunc}} = \sum_{k \neq k'} \gcd(k,k') /
(kk')$ contributes an $O(1)$ factor that makes $E_1^2 / (s \cdot E_2) < 1$
for $s/N \geq 0.4$.

\begin{center}
\begin{tabular}{@{}rcccc@{}}
\toprule
$n$ & $s/N = 0.5$ (HC-adv) & $s/N = 0.7$ & $s/N = 0.9$ & Worst \\
\midrule
500 & 1.19 & 1.06 & 0.92 & 0.92 \\
1{,}000 & --- & --- & 0.91 & 0.91 \\
2{,}000 & 1.21 & --- & 0.91 & 0.84 \\
3{,}000 & 1.07 & --- & 0.96 & 0.86 \\
\bottomrule
\end{tabular}
\end{center}

The CS failure band (where $\CS < 1$) spans $s/N \in [0.29, 0.92]$
at $n = 3{,}000$ and \emph{grows} with~$n$.  The deepest failure
is at $s/N \approx 0.45$--$0.49$, reaching $\CS \approx 0.86$.

\paragraph{Verdict: DEAD.}
Standard CS cannot prove global Hall for large~$n$.

\subsection{Per-interval Cauchy--Schwarz}\label{subsec:per-interval-cs-detail}

As described in Section~\ref{subsec:per-interval-cs}, restricting
to a single dyadic interval $I_j$ makes $G_{\mathrm{trunc}} \to 0$
and hence $\CS(I_j) \to \infty$.  This \emph{works} within each interval.

\paragraph{Why it doesn't globalize.}
The neighborhoods $\NH(S \cap I_j)$ overlap by 87--93\% across intervals
(Section~\ref{subsec:overlap}).  Summing $\sum_j |\NH(S \cap I_j)|
\geq \sum_j |S \cap I_j| = |S|$ is useless because the left side
double-counts.

\paragraph{Verdict:} Proves per-interval Hall; useless for global Hall.

\subsection{Weighted Cauchy--Schwarz}\label{subsec:weighted-cs}

The generalized CS bound allows an arbitrary weight function $f\colon V \to
\mathbb{R}_{>0}$:
\[
  |\NH(S)| \;\geq\; \frac{\bigl(\sum_{k \in S} f(k) \deg(k)\bigr)^2}
  {\sum_{h} \bigl(\sum_{k \in S:\, k \mid h} f(k)\bigr)^2}
  \;=\; \frac{(d^\top f)^2}{f^\top C f},
\]
where $C$ is the codegree matrix.  The optimal weight is
$f^* = C^{-1} d$, giving $|\NH(S)| \geq d^\top C^{-1} d$.

\paragraph{Computational results (Z43).}
The optimal weighted CS ratio $d^\top C^{-1} d / |S|$ is remarkably
stable at $1.29$--$1.33$ across all tested adversarial subsets and
all~$n$ up to $50{,}000$.  Several explicit weight functions also work:

\begin{center}
\begin{tabular}{@{}lcc@{}}
\toprule
Weight $f(k)$ & Ratio at $n = 10{,}000$ & Status \\
\midrule
$f = 1$ (standard) & 0.85 & FAILS \\
$f = \deg(k)$ & 0.72 & FAILS \\
$f = 1/\bar{\tau}(k)$ & 1.05 & passes \\
$f = 1/\sqrt{C_{kk}}$ & 1.16 & passes \\
$f = 1/\bar{\mu}(k)$ & 1.20 & passes \\
$f = C^{-1}d$ (optimal) & 1.31 & passes \\
\bottomrule
\end{tabular}
\end{center}

The ``anti-codegree'' weights (downweighting heavily-shared elements)
consistently work.

\paragraph{Why it fails as a proof.}
To convert this into a proof, one would need an \emph{analytic} expression
for $f^*$ or a provable lower bound on $d^\top C^{-1} d$.  The matrix~$C$
is a dense, $|S| \times |S|$ matrix whose entries depend on the
number-theoretic structure of divisibility.  We found no way to bound
$d^\top C^{-1} d \geq |S|$ analytically.  The condition number of~$C$
grows from $8{,}000$ to $574{,}000$, and the optimal weights exhibit
complex, irregular patterns that defy closed-form description.

\paragraph{Verdict: DEAD} as a proof technique, but computationally
the optimal CS passes at every~$n$ tested.

\subsection{Filtered Cauchy--Schwarz}\label{subsec:filtered-cs}

\paragraph{Idea.}
Exclude high-codegree targets from the CS sum.  Define $H' = \{h \in H :
\tau_S(h) \leq \tau_0\}$ for a threshold~$\tau_0$.  Then
$|\NH(S)| \geq |H'|$ trivially, and the CS bound on~$H'$ has smaller~$E_2$.

\paragraph{Computational results (Z112k).}
Even with optimal threshold selection, the filtered CS ratio peaks
at~$\sim\!0.991$ for adversarial subsets---still below~$1$.
The problem is that filtering removes the very targets that contribute
most to $E_1$, and the improvement in~$E_2$ does not compensate.

\paragraph{Verdict: DEAD.} Ratio $\sim\!0.991 < 1$.

\subsection{Truncated Cauchy--Schwarz}\label{subsec:truncated-cs}

\paragraph{Idea.}
Restrict to the Ford divisor cap: only count divisors $d$ of~$h$ in
the range $[h^{1/u}, h^{1-1/u}]$ (the ``medium'' divisors),
applying Ford's theorem~\cite{Ford2008} on the concentration of divisors.

\paragraph{Computational results.}
The truncated codegree sum is smaller, but the edge count also drops.
The ratio lands at $0.41$--$0.49$---far from~$1$.

\paragraph{Verdict: DEAD.} Ratio $< 0.5$.

\subsection{Variable-constant Cauchy--Schwarz}\label{subsec:variable-cs}

\paragraph{Idea (Z114).}
Perhaps the Cauchy--Schwarz bound \emph{does} prove Hall, but only at a
constant $C_{\mathrm{mult}} > 1$ (i.e., with a slightly larger interval
length $L' = C_{\mathrm{mult}} \cdot L$).  If $C_{\mathrm{crit}}(n) \to 1$
as $n \to \infty$, this would still prove the upper bound with any~$\eps > 0$.

\paragraph{Computational results (Z114).}
A $11 \times 10$ grid sweep over $n \in \{2\text{K}, \ldots, 100\text{K}\}$
and $C_{\mathrm{mult}} \in \{1.00, 1.01, \ldots, 1.50\}$ reveals:

\begin{enumerate}
\item $C_{\mathrm{crit}}(n) = 1.00$ for $n \leq 7{,}000$ and
  $C_{\mathrm{crit}}(n) = 1.01$ for $n \geq 10{,}000$.
  The critical multiplier does \emph{not} decrease toward~$1$.
\item At fixed $C_{\mathrm{mult}} > 1$, $\CS(T_0)$ \emph{stabilizes}
  rather than growing:
  at $C_{\mathrm{mult}} = 1.01$, $\CS(T_0) = 1.001$ at $n = 100{,}000$
  (approaching~$1$ from above).
\item The adversarial subset fraction $|T_0|/|V|$ \emph{grows} from
  $45\%$ to $53\%$ as $n$ increases.
\end{enumerate}

\begin{figure}[ht]
\centering
\includegraphics[width=\textwidth]{fig_variable_constant.pdf}
\caption{Variable-constant CS: $\CS(T_0)$ vs.~$n$ for six values of
  $C_{\mathrm{mult}}$.  At $C_{\mathrm{mult}} = 1.00$ (dashed), the
  ratio oscillates below~$1$.  At larger multipliers, the ratio
  stabilizes but does not grow.}
\label{fig:variable-constant}
\end{figure}

\paragraph{Verdict: DEAD.}
$C_{\mathrm{crit}}$ does not converge to~$1$.  At $C_{\mathrm{mult}} = 1.01$,
the CS ratio approaches~$1$ from above and may eventually cross below.
The variable-constant approach cannot close the gap.

%% ===================================================================
\section{Matching and fractional methods}\label{sec:matching}
%% ===================================================================

This section documents eight approaches based on constructing matchings
(exact or fractional) in the bipartite graph~$G_n$.  The common theme:
matchings exist (HK confirms this), but no constructive or
fractional argument can \emph{prove} existence for all~$n$.

\subsection{Greedy matching}\label{subsec:greedy}

\paragraph{Method.}
Match elements in order of increasing degree (hardest first).  Each
element~$k$ is assigned to its available multiple with smallest
$\tau_S(m)$ (least contested).

\paragraph{Results.}
Greedy matching fails in 112 out of 220 tested configurations:

\begin{center}
\begin{tabular}{@{}rccc@{}}
\toprule
$n$ & tests & failures & failure rate \\
\midrule
500 & 55 & 20 & 36\% \\
1{,}000 & 55 & 24 & 44\% \\
2{,}000 & 55 & 33 & 60\% \\
3{,}000 & 55 & 35 & 64\% \\
\bottomrule
\end{tabular}
\end{center}

\paragraph{Why it fails.}
Elements with degree~$\sim\!3$--$8$ (e.g., $k = 98 = 2 \cdot 7^2$,
$k = 81 = 3^4$) get stuck because \emph{all} their multiples were
claimed by previously matched degree-$2$ elements.  The greedy makes
locally optimal but globally suboptimal choices.  The failure rate
\emph{grows} with~$n$.

\paragraph{Verdict: DEAD.}

\subsection{Uniform fractional matching}\label{subsec:uniform-frac}

\paragraph{Method.}
Assign weight $x_{k,h} = 1/\tau_S(h)$ to each edge.  The right-side
constraint $\sum_k x_{k,h} = 1$ is automatic.  If the left-side weight
$w(k) = \sum_{h:\, k \mid h} 1/\tau_S(h) \geq 1$ for all~$k$, then
Birkhoff--von~Neumann integrality gives a perfect matching.

\paragraph{Results.}
The condition $w(k) \geq 1$ for all~$k$ is \textbf{false}.  The
minimum $w(k)$ ranges from $0.057$ to $0.087$---far below~$1$.
The worst element~$k^*$ is always a highly composite number near~$N$
with $\deg(k^*) \approx \de \approx 2$--$3$ and all multiples having
$\tau_S(h) \gg 20$.  The uniform weight $1/\tau_S(h)$ starves
large elements: $w(k^*) \approx \deg / \bar{\tau} \approx 2/30
\approx 0.07$.

\paragraph{Verdict: DEAD.}

\subsection{Nonuniform fractional matching}\label{subsec:nonuniform-frac}

\paragraph{Method.}
Solve the LP: find $x_{k,h} \geq 0$ with $\sum_h x_{k,h} \geq 1$
for all~$k$ and $\sum_k x_{k,h} \leq 1$ for all~$h$.  This always
has an optimal solution (since a perfect matching exists), but the
fractional load $\max_h \sum_k x_{k,h}$ may hover near~$1.0$.

\paragraph{Results.}
LP-optimal fractional matchings exist with $\max$ load $= 1.0$
(tight).  However, the dual LP reveals that the bottleneck elements
are exactly the min-degree vertices near~$N$, and the fractional
solution routes flow through heavily shared targets---reflecting the
same structural difficulty.

\paragraph{Verdict: DEAD} as a proof technique (existence of LP
optimum does not constitute a Hall proof for all~$n$).

\subsection{FMC with dyadic intervals}\label{subsec:fmc-dyadic}

\paragraph{Method.}
Partition $V$ into $J \approx \frac{1}{2}\log_2 n$ dyadic intervals
and apply the FMC theorem (Theorem~\ref{thm:fmc}).

\paragraph{Results (Z99).}
The FMC sum $\Sigma = \sum_j 1/\alpha_j$ is:

\begin{center}
\begin{tabular}{@{}rcc@{}}
\toprule
$n$ & $\Sigma$ (greedy) & Status \\
\midrule
500 & 1.109 & FAIL \\
1{,}000 & 1.154 & FAIL \\
5{,}000 & 1.071 & FAIL \\
10{,}000 & 1.016 & FAIL \\
20{,}000 & 0.917 & OK \\
50{,}000 & 0.755 & OK \\
\bottomrule
\end{tabular}
\end{center}

The FMC sum (using greedy $\alpha_j$) crosses below~$1$ at
$n \approx 15{,}000$--$20{,}000$.  The CS-based overestimate
$\sum 1/\CS_j$ is consistently larger than the greedy-based sum
$\sum 1/\alpha_j$, illustrating that CS underestimates
the true expansion within intervals.

\paragraph{Why it fails for small $n$.}
The bottom interval (vertices near~$N$) has $\alpha_j \approx d_{\min}
\approx 3$, contributing $\sim\!0.33$ to the FMC sum.  With
$J \approx 7$--$10$ intervals, even small contributions from other
intervals push the total above~$1$.

\paragraph{Verdict: DEAD} for $n \lesssim 15{,}000$; works computationally
but not provably for larger~$n$.

\subsection{Three-block FMC}\label{subsec:three-block}

\paragraph{Method (Z103).}
Partition $V = R \cup S_+ \cup S_-$ where $R$ = rough numbers ($P(k) > B$),
$S_+ = \{k > B : P(k) \leq B\}$ (smooth, large), and $S_- = \{1,\ldots,B\}$
(smooth, small).  The three-block FMC needs
$1/\alpha(R) + 1/\alpha(S_+) + 1/\alpha(S_-) < 1$.

\paragraph{Results (Z103b).}

\begin{center}
\begin{tabular}{@{}rccccl@{}}
\toprule
$n$ & $\alpha(R)$ & $\alpha(S_+)$ & $\alpha(S_-)$ & FMC & Status \\
\midrule
1{,}000 & 2.00 & 1.31 & 27.3 & 1.302 & FAIL \\
5{,}000 & 2.40 & 1.59 & 75.4 & 1.059 & FAIL \\
10{,}000 & 2.50 & 2.30 & 117.9 & 0.843 & OK \\
50{,}000 & 3.00 & 3.00 & 350.9 & 0.670 & OK \\
\bottomrule
\end{tabular}
\end{center}

The three-block approach improves over dyadic (crossover at $n \approx
10{,}000$ vs $20{,}000$), but $\alpha(S_+)$ remains unproved analytically.
The greedy heuristic overestimates $\alpha$ by 10--15\%, so the true crossover
may be later.

\paragraph{Verdict: DEAD} without an analytic bound on $\alpha(S_+)$.

\subsection{Sinkhorn iteration}\label{subsec:sinkhorn}

\paragraph{Method.}
Apply Sinkhorn's algorithm (alternating row/column normalization) to the
biadjacency matrix to find a doubly stochastic scaling.  If the scaling
converges, Birkhoff's theorem guarantees a perfect matching.

\paragraph{Results (Z113c).}
Sinkhorn converges, but the fractional deficiency (deviation from doubly
stochastic) is 13--15\%.  The minimum row sum after $1{,}000$ iterations
is $0.85$--$0.87$, not reaching~$1$.  The bottleneck is again the
min-degree vertices near~$N$.

\paragraph{Verdict: DEAD.}

\subsection{CLP factoring}\label{subsec:clp}

\paragraph{Method (Z113c).}
Factor the bipartite graph into perfect matchings via the constructive
Lov\'asz--Plummer algorithm.  If $G_n$ has a decomposition into~$\de$
edge-disjoint perfect matchings, the first one suffices.

\paragraph{Results.}
$G_n$ is not regular ($\deg(k)$ varies from~$\de$ to~$M$), so direct
factoring fails.  Even after truncating to the $\de$-regular subgraph
(keeping only $\de$ edges per vertex), the resulting graph is not
bipartite-regular and does not decompose cleanly.

\paragraph{Verdict: DEAD.}

\subsection{$\sqrt{2}$-partition sequential matching}\label{subsec:sqrt2}

\paragraph{Method (Z31g).}
Partition $V$ using ratio $\sqrt{2}$ (intervals $[c^j, c^{j+1})$ with
$c = \sqrt{2}$) instead of ratio~$2$.  Within each finer interval, the
local degree condition $\min_{k \in I_j} \deg(k) \geq
\max_{h \in \NH(I_j)} |\{k \in I_j : k \mid h\}|$ holds, enabling
sequential matching interval by interval.

\paragraph{Results.}
At every tested~$n$ up to $200{,}000$, the $\sqrt{2}$-partition satisfies
the local degree condition in every interval (worst ratio $= 1.00$).
The number of intervals grows from~$7$ to~$14$.

\paragraph{Why it fails as a proof.}
Sequential matching across intervals requires that targets used by
earlier intervals remain available for later intervals.  The 87--93\%
overlap means most targets are shared across intervals, and there is no
guarantee that a greedy interval-by-interval approach leaves enough
targets for subsequent intervals.  This is the same globalization
barrier as in Section~\ref{subsec:per-interval-cs-detail}.

\paragraph{Verdict: DEAD} for the global argument; the local condition
passes everywhere.

%% ===================================================================
\section{Probabilistic methods}\label{sec:probabilistic}
%% ===================================================================

Probabilistic existence arguments are among the most powerful tools
in combinatorics.  We tried five probabilistic approaches; all fail
because the bipartite graph~$G_n$ has $D \approx D_2$ (degree~$\approx$
maximum codegree for the hard vertices), precisely the regime where
these methods lose power.

\subsection{Symmetric Lov\'asz Local Lemma}\label{subsec:symmetric-lll}

\paragraph{Method.}
Random matching: each $k \in V$ picks $\varphi(k)$ uniformly from its
$\deg(k)$ multiples in~$H$.  The bad event $B_k$ is a collision
($\varphi(k) = \varphi(k')$ for some $k' \neq k$).  The symmetric
LLL~\cite{ErdosSpencer1991} asserts that if $\Pr[B_k] \cdot e \cdot
(d_k + 1) \leq 1$ for all~$k$, where $d_k$ is the number of other
bad events sharing a target with~$B_k$, then a proper injection exists
with positive probability.

\paragraph{Results.}
The symmetric LLL works in only 4 out of 108 tested configurations
(all trivial: top-packed subsets at $s/N = 0.1$).

\begin{center}
\begin{tabular}{@{}rccc@{}}
\toprule
$n$ & $\max \Pr[B_k]$ & $\max d_k$ & $\max$ LLL value \\
\midrule
500 & 1.83 & 224 & 540 \\
1{,}000 & --- & --- & $>1{,}000$ \\
2{,}000 & 2.13 & 868 & 2{,}091 \\
\bottomrule
\end{tabular}
\end{center}

\paragraph{Why it fails.}
Three compounding problems:
(i)~$\Pr[B_k] > 1$ is not meaningful (collision is near-certain under
uniform random matching for hard elements);
(ii)~the dependency degree $d_k \approx s$ for small elements like
$k = 6, 12$ (they share multiples with essentially every other element);
(iii)~the product $\Pr[B_k] \cdot d_k$ \emph{grows} with~$n$.

\paragraph{Verdict: DEAD.}

\subsection{Target-centered LLL}\label{subsec:target-lll}

\paragraph{Method.}
Define bad events on targets instead: $B_h = \{|\{k : \varphi(k) = h\}|
\geq 2\}$ (target~$h$ receives two or more elements).  The dependency
graph is sparser (each $B_h$ depends only on $B_{h'}$ where $h$ and $h'$
share a source).

\paragraph{Results.}
The target-centered formulation has $\Pr[B_h] \leq 1 - (1 - 1/\bar{d})^{\tau(h)}$
$\approx \tau(h)/\bar{d}$, and the dependency degree is $d_h \leq
\sum_{k:\,k \mid h} (\deg(k) - 1)$.  For highly composite targets
($\tau(h) > 50$), we get $\Pr[B_h] \approx 1$, and the LLL condition
$\Pr[B_h] \cdot e \cdot (d_h + 1) \leq 1$ fails catastrophically.

\paragraph{Feasibility check (Z113a).}
$P \cdot e \cdot (D + 1) \approx 10^4$ at $n = 50{,}000$---four orders of
magnitude above the threshold.

\paragraph{Verdict: DEAD.}

\subsection{Janson's inequality}\label{subsec:janson}

\paragraph{Method.}
Janson's inequality~\cite{Janson1998} gives a bound on the probability
that a random subset avoids all ``bad configurations.''  Applied to our
setting with random target assignment, the Janson bound requires
the sum of pairwise dependencies $\Delta = \sum_{(B_i, B_j) \text{ dep.}}
\Pr[B_i \wedge B_j]$ to be small relative to $(\sum \Pr[B_i])^2$.

\paragraph{Results.}
$\Delta \gg (\sum \Pr[B_i])^2$ because the codegree structure is too
dense.  For vertices near~$N$ with $\deg \approx \de \approx 3$,
every pair of targets of~$k$ is shared with other vertices, making
the joint probabilities comparable to the marginal probabilities.

\paragraph{Verdict: DEAD.}

\subsection{Erd\H{o}s--Spencer weighted LLL}\label{subsec:erdos-spencer}

\paragraph{Method.}
The Erd\H{o}s--Spencer~\cite{ErdosSpencer1991} version of the LLL allows
non-uniform probabilities: assign $x_i \in (0, 1)$ to each bad event
$B_i$ such that $\Pr[B_i] \leq x_i \prod_{j \sim i} (1 - x_j)$.

\paragraph{Results.}
The optimal $x_i$ must satisfy a system of $|V|$ nonlinear inequalities.
For our graph, the system is infeasible: the high-codegree vertices
force $x_i \to 1$, which makes $\prod (1 - x_j) \to 0$, creating a
circular obstruction.

\paragraph{Verdict: DEAD.}

\subsection{Semi-random nibble}\label{subsec:nibble}

\paragraph{Method (Z115).}
R\"odl nibble / semi-random process~\cite{Rodl1985,MolloyReed2002}:
each round, every unmatched vertex picks a random available target;
if a target is picked by exactly one vertex, match them.  Repeat
for 500~rounds.  Run HK on the residual graph to check matchability.

\paragraph{Results.}

\begin{center}
\begin{tabular}{@{}rccc@{}}
\toprule
$n$ & avg.\ residual \% & HK on residual & unmatchable / $|V|$ \\
\midrule
1{,}000 & 52.2\% & FAILS & 29.7\% \\
5{,}000 & 43.3\% & FAILS & --- \\
10{,}000 & 38.7\% & FAILS & 29.4\% \\
20{,}000 & 34.4\% & FAILS & --- \\
30{,}000 & 32.3\% & FAILS & 24.8\% \\
50{,}000 & 29.5\% & (not tested) & --- \\
\bottomrule
\end{tabular}
\end{center}

\paragraph{Why it fails.}
The nibble matches $\sim\!65$--$70\%$ of vertices, but the remaining
$30$--$35\%$ include $\sim\!25\%$ of~$|V|$ that are \emph{genuinely
unmatchable} in the residual graph.  HK fails on every residual at every~$n$
and every random seed (variance across seeds is $\pm 1$--$2\%$---the failure
is structural, not bad luck).

\paragraph{Theoretical diagnosis.}
The R\"odl nibble theory requires $D / D_2 \to \infty$ (degree~$\gg$
max codegree).  For our graph, $D \approx D_2 \approx \de$ for the
hard vertices near~$N$.  The theory correctly predicts failure.

When a hard vertex~$k_1$ ($\deg \approx 3$--$5$) randomly claims
target~$h$, other hard vertices~$k_2$ that also needed~$h$ (as one of
their few options) permanently lose a critical target.  After enough
rounds, many vertices have lost \emph{all} viable targets.

\begin{figure}[ht]
\centering
\includegraphics[width=\textwidth]{fig_nibble_residual.pdf}
\caption{Semi-random nibble results.  The residual fraction (blue)
  shrinks with~$n$, but the permanently unmatchable fraction (red)
  remains at $25$--$30\%$ of~$|V|$.  HK fails on every residual
  at every~$n$ tested.}
\label{fig:nibble}
\end{figure}

\paragraph{Verdict: DEAD.}
This is approach \#43 on the dead list.

%% ===================================================================
\section{Sieve and inclusion-exclusion methods}\label{sec:sieve}
%% ===================================================================

Sieve methods estimate $|\NH(S)|$ by subtracting the number of targets
\emph{not} hit by any element of~$S$.  Inclusion-exclusion gives an
exact formula, but its partial sums (Bonferroni bounds) diverge for
our graph.

\subsection{Bonferroni bounds (all orders)}\label{subsec:bonferroni}

\paragraph{Method.}
The exact neighborhood size is
\[
  |\NH(S)| = \sum_{h \in H} \mathbf{1}[\tau_S(h) \geq 1]
  = \sum_{t=1}^{|S|} (-1)^{t+1} \binom{|S|}{t}^{-1}
    \sum_{T \subseteq S,\, |T|=t} |\NH(T)|.
\]
The first-order (union) bound gives $|\NH(S)| \leq E_1$.  The
second-order Bonferroni bound subtracts pairwise overlaps:
\[
  |\NH(S)| \geq E_1 - \textstyle\sum_{k < k'} \deg(k, k')
  = E_1 - \text{codegree sum}.
\]
For this to prove Hall ($|\NH(S)| \geq s$), we need the codegree sum
$\leq E_1 - s = \text{``excess.''}$

\paragraph{Results (Z07).}
The Bonferroni bound fails in \emph{all} 80~tested configurations at
\emph{all} values of~$n$.  The ``waste ratio'' (codegree sum / true overlap)
ranges from $2.4$ to $9.2$ and \emph{grows} with~$n$:

\begin{center}
\begin{tabular}{@{}rcc@{}}
\toprule
$n$ & min waste ratio & max waste ratio \\
\midrule
500 & 2.39 & 5.72 \\
1{,}000 & 2.32 & 6.61 \\
2{,}000 & 2.39 & 7.69 \\
3{,}000 & 3.02 & 8.36 \\
5{,}000 & 3.11 & 9.22 \\
\bottomrule
\end{tabular}
\end{center}

\paragraph{Root cause.}
The codegree sum counts \emph{pairs} sharing a target: if a target~$h$
has $\tau = \tau_S(h)$ divisors from~$S$, the codegree sum gets
$\binom{\tau}{2}$, but the true overlap is only $\tau - 1$.  The ratio
$\binom{\tau}{2} / (\tau - 1) = \tau / 2$ grows with~$\tau$.  Since
many targets have $\tau \geq 5$--$30$, the Bonferroni bound is
quadratically wasteful.

Higher-order Bonferroni terms ($t = 3, 4, \ldots$) alternate in sign
and grow even faster, so the series does not truncate usefully.

\paragraph{Verdict: DEAD.}

\subsection{Product-formula sieve}\label{subsec:product-sieve}

\paragraph{Method.}
The independent sieve estimate: the number of targets hit by \emph{no}
element of~$S$ is approximately
$\text{sifted} \approx M \cdot \prod_{k \in S}(1 - 1/k) \approx M
e^{-\sigma}$, where $\sigma = \sum_{k \in S} 1/k$.  Then
$|\NH(S)| \approx M - M e^{-\sigma} = M(1 - e^{-\sigma})$.

\paragraph{Results (Z08).}
The actual sifted count is \emph{always larger} than the independent
estimate (ratio $2$--$300\times$).  The divisibility events $k_1 \mid h$
and $k_2 \mid h$ are \emph{positively correlated} when $\gcd(k_1, k_2)
> 1$, so more targets survive sifting than independence predicts.
The product formula \emph{overestimates} $|\NH(S)|$ and bounds in the
wrong direction.

\paragraph{Verdict: DEAD.}  Gives upper, not lower, bound.

\subsection{Unique multiple sieve}\label{subsec:unique-sieve}

\paragraph{Method.}
Show every $k \in S$ has at least one \emph{unique} multiple
$h \in H$ (with $\tau_S(h) = 1$).  This immediately gives
$|\NH(S)| \geq |S|$.

\paragraph{Results.}
Even for top-packed subsets, elements with $d_1(k) = 0$ (no unique
multiples) appear at $s/N \geq 0.3$.  For adversarial subsets,
$30$--$90\%$ of elements have no unique multiples.  The elements with
$d_1 = 0$ are large~$k$ near~$N$ whose few multiples are all shared.

\paragraph{Verdict: DEAD} for general subsets.

\subsection{Standard sieve}\label{subsec:standard-sieve}

\paragraph{Method.}
Apply the Selberg sieve or Brun sieve to estimate the number of
integers in~$H$ not divisible by any $k \in S$.

\paragraph{Results.}
The standard sieve gives a safety factor of $\sim\!3.56$ (the sifted
count is $3.56\times$ larger than the independent estimate), but in the
wrong direction: it \emph{overestimates} the sifted count, meaning it
\emph{underestimates} $|\NH(S)|$.  The sieve's systematic error comes
from the same positive correlations noted above.

\paragraph{Verdict: DEAD.}

%% ===================================================================
\section{Graph-theoretic and spectral methods}\label{sec:graph-theoretic}
%% ===================================================================

This section covers approaches based on the graph structure of~$G_n$
or its associated conflict graph, rather than on direct counting arguments.

\subsection{Spectral gap}\label{subsec:spectral}

\paragraph{Method.}
The expander mixing lemma: if $G_n$ has adjacency matrix with second
singular value~$\sigma_2$ and largest singular value~$\sigma_1$,
then for any $S \subseteq V$:
\[
  |\NH(S)| \;\geq\; \frac{|S| \cdot |H| \cdot \bar{d}^2}
  {|S| \cdot \bar{d}^2 + \sigma_2^2 \cdot |H|},
\]
which exceeds~$|S|$ when $\sigma_2 / \sigma_1$ is small (i.e.,
the graph is a good expander).

\paragraph{Results.}
The spectral gap $\sigma_2 / \sigma_1$ is too large.  The bipartite
graph~$G_n$ is far from regular (degree varies from~$\de$ to~$M$),
so the spectrum is dominated by the high-degree vertices.  The
spectral bound gives $|\NH(S)| \geq C \cdot |S|$ only for
$C \ll 1$---useless for Hall's condition.

\paragraph{Verdict: DEAD.}

\subsection{Haxell's independent transversal theorem}\label{subsec:haxell}

\paragraph{Method.}
Haxell's theorem~\cite{Haxell1995}: if a bipartite graph $G = (A, B, E)$
is partitioned into blocks $A = A_1 \cup \cdots \cup A_r$ and the
``conflict graph'' on targets has maximum degree~$\Delta$, then a
system of distinct representatives exists provided
$|A_i| \geq 2\Delta$ for all~$i$.

\paragraph{Results.}
The conflict graph on~$H$ (where two targets are adjacent if they share
a source) has high maximum degree because highly composite targets
are connected to many others.  The condition $|A_i| \geq 2\Delta$ fails
for the bottom intervals where $|A_i|$ is small and $\Delta$ is large.
The cross-interval overlap of 87--93\% means the conflict graph is nearly
complete.

\paragraph{Verdict: DEAD.}

\subsection{Tur\'an / maximum weighted independent set}\label{subsec:turan}

\paragraph{Method.}
Hall's condition can be rephrased via the LCM conflict graph:
$k_1 \sim k_2$ if $\lcm(k_1, k_2) \leq n + L$ (i.e., they share a
target).  A maximum weighted independent set (MWIS) of size~$\geq s$
in $G_n[S]$ gives disjoint neighborhoods, implying $|\NH(S)| \geq s$.

The Tur\'an bound gives $\alpha(G) \geq |V|^2 / (|V| + 2|E|)$.

\paragraph{Results.}
The Tur\'an bound gives only 27--32\% of~$|S|$, and the ratio
\emph{decreases} with~$n$:

\begin{center}
\begin{tabular}{@{}rcc@{}}
\toprule
$n$ & Tur\'an / $|S|$ & Greedy IS / $|S|$ \\
\midrule
500 & 0.324 & 0.302 \\
1{,}000 & 0.308 & 0.304 \\
5{,}000 & 0.279 & 0.300 \\
10{,}000 & 0.269 & 0.298 \\
\bottomrule
\end{tabular}
\end{center}

The greedy independent set is $10$--$27\times$ larger than the Tur\'an
bound, and this gap \emph{grows} with~$n$.  Generic graph bounds cannot
capture the arithmetic structure that makes the actual independent set large.

\paragraph{Verdict: DEAD.}

\subsection{Degeneracy}\label{subsec:degeneracy}

\paragraph{Method.}
The degeneracy~$d$ of the conflict graph gives a coloring bound
$\chi \leq d + 1$, and hence $\alpha \geq |V| / (d + 1)$.

\paragraph{Results.}
Degeneracy grows as $\sim\!15\sqrt{n}$:

\begin{center}
\begin{tabular}{@{}rcc@{}}
\toprule
$n$ & degeneracy & $|V| / (d+1)$ as fraction of target \\
\midrule
500 & 31 & 28\% \\
1{,}000 & 46 & 22\% \\
5{,}000 & 105 & 14\% \\
10{,}000 & 151 & 10\% \\
\bottomrule
\end{tabular}
\end{center}

The degeneracy bound gives only 10--28\% of what is needed, and the
fraction \emph{decreases} with~$n$.

\paragraph{Verdict: DEAD.}

\subsection{Ford divisor cap}\label{subsec:ford-cap}

\paragraph{Method.}
Apply Ford's theorem~\cite{Ford2008} on the distribution of integers
with a divisor in a given interval to bound the number of ``medium''
divisors of targets in~$H$.

\paragraph{Results.}
The Ford divisor cap restricts $\tau_S(h)$ to the range
$[h^{1/u}, h^{1-1/u}]$, giving bounds of $0.41$--$0.49$ on the
truncated CS ratio.

\paragraph{Verdict: DEAD.}  Ratio $< 0.5$.

\subsection{Multiplicative energy (Koukoulopoulos--Maynard)}\label{subsec:mult-energy}

\paragraph{Method (Z44).}
The Koukoulopoulos--Maynard GCD graph technique~\cite{KoukoulopoulosMaynard2020}
decomposes sets into ``structured'' (high multiplicative energy) and
``unstructured'' (low energy) parts.  The structured part can be handled
by density-increment arguments, and the unstructured part has good expansion.

\paragraph{Results.}
The adversarial set~$T$ has multiplicative energy $E_{\mathrm{mult}} / |T|^2
\in [2.0, 7.8]$, which is mildly structured at small~$n$ but collapses to
the trivial diagonal contribution ($\approx 2.0$) at $n = 50{,}000$.
The adversary becomes multiplicatively \emph{independent} at large~$n$:
it lives in the ``unstructured regime'' where the K--M expansion argument
should work.

\paragraph{Why it doesn't close the gap.}
The K--M framework gives qualitative expansion ($|\NH(S)| \geq
(1 + c)|S|$ for some $c > 0$), but the constant~$c$ depends on the
specific GCD graph parameters and has not been made explicit enough to
verify $c > 0$ for the Erd\H{o}s~710 graph.  Moreover, the framework
was designed for the Duffin--Schaeffer conjecture, where the bipartite
graph is denser and more regular than ours.

\paragraph{Verdict:} Promising direction but not yet converted to a
proof.  The multiplicative energy analysis shows the adversary is
\emph{not} exploiting product structure---its power comes from the
``diffuse, low-degree'' nature of the graph, which is harder to handle.

%% ===================================================================
\section{Interesting phenomena}\label{sec:phenomena}
%% ===================================================================

Beyond the proof attempts, our computational investigation uncovered
several striking phenomena that illuminate the structure of the
Erd\H{o}s~710 bipartite graph.

\subsection{Sawtooth oscillations}\label{subsec:sawtooth}

\begin{observation}[Z65]\label{obs:sawtooth}
The FMC sum $\Sigma(n) = \sum_{j=1}^{J(n)} 1/\alpha_j$ exhibits a
\emph{sawtooth} pattern as a function of~$n$, with sharp drops at values
of~$n$ where the number of dyadic intervals $J(n)$ increases by~$1$.
\end{observation}

This is perhaps the most visually striking phenomenon in the entire
investigation, and it reveals the mechanism by which the problem
``breathes'' as~$n$ grows.

\paragraph{The mechanism.}
Within a fixed $J$-regime (fixed number of dyadic intervals), the FMC
sum rises steadily: as~$n$ increases, new smooth numbers enter the
bottom interval, the codegree accumulates, and the expansion ratio
$\alpha_j$ of the bottleneck interval slowly deteriorates.  Then,
at a critical value of~$n$, the number of intervals jumps
$J \to J + 1$: the hardest interval gets split in two, each with
better degree homogeneity.  The FMC sum drops sharply---a
``shockwave''---and the cycle begins again.

The transitions are non-monotonic.  At some values of~$n$, $J$
temporarily \emph{decreases} (an interval becomes empty as the
smoothness bound~$B$ shifts), causing an upward shock:

\begin{center}
\begin{tabular}{@{}ccl@{}}
\toprule
$n$ & transition & direction \\
\midrule
140 & $J\!: 2 \to 3$ & $\uparrow$ new interval \\
280 & $J\!: 4 \to 3$ & $\downarrow$ interval empties \\
520 & $J\!: 3 \to 4$ & $\uparrow$ \\
2{,}050 & $J\!: 4 \to 5$ & $\uparrow$ \\
8{,}200 & $J\!: 5 \to 6$ & $\uparrow$ \\
34{,}500 & $J\!: 6 \to 7$ & $\uparrow$ \\
68{,}500 & $J\!: 7 \to 8$ & $\uparrow$ \\
266{,}000 & $J\!: 8 \to 9$ & $\uparrow$ \\
\bottomrule
\end{tabular}
\end{center}

\paragraph{The peak envelope.}
The peak of each $J$-regime defines an envelope tracking the
worst-case FMC sum:

\begin{center}
\begin{tabular}{@{}rcc@{}}
\toprule
$J$ & peak $n$ & peak $\Sigma$ \\
\midrule
4 & 520 & 0.793 \\
5 & 2{,}050 & 0.850 \\
6 & 8{,}200 & 0.831 \\
7 & 34{,}500 & 0.857 \\
8 & 68{,}500 & \textbf{0.881} \\
9 & 266{,}000 & 0.871 \\
\bottomrule
\end{tabular}
\end{center}

The global maximum over all 461~data points is $\Sigma = 0.881$
at $n = 68{,}500$ (within the $J = 8$ regime), leaving an $11.9\%$
margin to the threshold~$1$.  The envelope is \emph{non-monotonic}:
$J = 6$ has a lower peak than $J = 5$, and $J = 9$ lower than $J = 8$.
This irregular growth makes extrapolation to large~$n$ unreliable---we
cannot determine whether the envelope eventually reaches~$1$ or
stabilizes below it.

\paragraph{The companion curves.}
Figure~\ref{fig:sawtooth} shows two curves:
$\sum 1/\CS_{\mathrm{ref},j}$ (the CS-based upper bound, blue) and
$\sum 1/\alpha_j$ (the greedy expansion ratio, green).  The greedy
curve tracks below the CS curve, confirming that the true expansion
is better than CS predicts.  The bottom panel shows $\delta(n)$
(orange, growing sublogarithmically) and $J(n)$ (purple, step function),
revealing the $J$-transitions that drive the sawtooth.

\begin{figure}[ht]
\centering
\includegraphics[width=\textwidth]{fig_sawtooth.pdf}
\caption{The FMC sawtooth.  \textbf{Top:} FMC sum
  $\sum_j 1/\alpha_j$ (green) and $\sum_j 1/\CS_{\mathrm{ref},j}$
  (blue) vs.~$n$ on a log scale, from 461~data points.
  Sharp drops (``shockwaves'') occur at $J$-transitions where the
  number of dyadic intervals changes.  The red dashed envelope tracks
  the peaks, reaching a global max of $0.881$ at $n = 68{,}500$.
  The green shaded region shows the $11.9\%$ margin to the
  threshold~$1$.
  \textbf{Bottom:} The minimum-degree parameter $\delta(n)$ (orange)
  and the number of intervals $J(n)$ (purple step function).}
\label{fig:sawtooth}
\end{figure}

\subsection{CS deficiency oscillation}\label{subsec:cs-oscillation}

\begin{observation}[Z113b, Z114]\label{obs:cs-deficiency}
The global Cauchy--Schwarz ratio $\CS(T_0)$ for the adversarial subset
oscillates near~$1.0$ as $n$ increases, without converging in either
direction.
\end{observation}

At $C_{\mathrm{mult}} = 1.00$ (the base constant), $\CS(T_0)$ crosses
above~$1$ at some~$n$ values and below~$1$ at others.  The oscillation
appears to be driven by the same sawtooth mechanism: the adversarial
subset~$T_0$ restructures when the number of dyadic intervals changes.

This non-monotonic behavior is one reason the variable-constant approach
(Section~\ref{subsec:variable-cs}) fails: there is no smooth function
$C_{\mathrm{crit}}(n) \to 1$ to track.

\subsection{GCD stratum decomposition}\label{subsec:gcd-stratum}

\begin{observation}\label{obs:gcd-stratum}
The codegree sum (``off-diagonal'' part of~$E_2$) is dominated by
pairs with moderate GCD: the stratum $\gcd(k_1, k_2) \in [11, 500]$
accounts for $80$--$95\%$ of the truncated codegree sum, while coprime
pairs ($\gcd = 1$) contribute only $0$--$2\%$.
\end{observation}

The concentration in moderate GCDs reflects the ``anatomy of smooth
numbers'': elements of the adversarial set tend to share $2$--$3$ small
prime factors (giving $\gcd \in [6, 500]$), which is enough to create
codegree without making the lcm exceed the truncation threshold.

Interestingly, the lcm truncation (excluding pairs with
$\lcm(k_1, k_2) > n + L$) removes only $3$--$11\%$ of the codegree sum.
The truncation is nearly invisible above $\gcd = 10$.

\subsection{Far-partner stability}\label{subsec:far-partner}

\begin{observation}\label{obs:far-partner}
The ``far-partner fraction''---the fraction of vertices~$k$ whose
codegree partner~$k'$ with $\lcm(k, k') > n + L$ contributes to
the harmonic sum---stabilizes at $8$--$13\%$ and does \emph{not}
decay to~$0$ with~$n$.
\end{observation}

This means that $\sim\!10\%$ of all vertices are ``forced far'' by the
product constraint, creating a persistent structural contribution to the
codegree.  The far-partner mechanism is driven by coprime elements
($\gcd(k, k') = 1$) at opposite ends of the interval.

\subsection{Shearer vs.\ CS dichotomy}\label{subsec:shearer-dichotomy}

\begin{observation}[Initial attempt]\label{obs:shearer}
Shearer's entropy method~\cite{ErdosSpencer1991} \emph{never fails} on
the Erd\H{o}s~710 graph: the Shearer bound gives $|\NH(S)| \geq C |S|$
with $C > 1$ at every tested~$n$ and every subset type.  Meanwhile,
Cauchy--Schwarz fails on adversarial subsets.
\end{observation}

The Shearer bound used here is
$|\NH(S)| \geq \bigl(\prod_{k \in S} \deg(k)\bigr)^{1/\Delta}$,
where $\Delta = \max_{h \in \NH(S)} \tau_S(h)$ is the maximum
right-degree.  Taking logarithms, this becomes
$\log|\NH(S)| \geq (1/\Delta) \sum_{k \in S} \log \deg(k)$.
This bound is stronger than CS when the multiplicity distribution
is concentrated (many targets with $\tau = 1$--$2$), which is
exactly the case for our graph.

However, Shearer's bound cannot be made into a proof either: it requires
\emph{computing} $H(\tau)$ for specific subsets, and no analytic estimate
of $H(\tau)$ is tight enough.

\subsection{LCM independent set margin}\label{subsec:lcm-margin}

\begin{observation}[Z09]\label{obs:lcm-margin}
The LCM independent set (vertices with pairwise $\lcm > n + L$) always
covers at least $|V|$ targets, with the tightest margin being
$|\NH(I_{\mathrm{LCM}})| / |I_{\mathrm{LCM}}| = 1.012$.
\end{observation}

This means that even after restricting to a maximal set of
``non-overlapping'' vertices, the target count barely exceeds the
vertex count.  The $1.2\%$ margin underscores the razor-thin expansion.

%% ===================================================================
\section{Results from the initial proof attempt}\label{sec:initial}
%% ===================================================================

Before the systematic investigation documented in
Sections~\ref{sec:cs-family}--\ref{sec:graph-theoretic}, an initial
proof attempt (Sessions~1--9) established several structural results
and identified the core difficulty.  We summarize these here both for
completeness and because some ideas remain promising.

\subsection{Proof reduction: $U/V$ split and Case~A/B}

The initial attempt established the same proof architecture described
in Sections~\ref{sec:construction}--\ref{subsec:cs-framework}:
\begin{enumerate}
\item Top half~$U$ matched by doubling ($k \mapsto 2k$).
\item Bottom half~$V$ requires Hall's condition in $G_n = (V, H, E)$.
\item Case~A ($\min(S) \leq M/(s+1)$) handled by single-element degree.
\item Case~B ($\min(S) > M/(s+1)$) is the remaining gap.
\end{enumerate}

Additionally, the initial attempt introduced a finer decomposition
using a ``smooth set'' $A = \{k \in [cn, n] : P^+(k) \leq B\}$
with $c = e^{-1/(2u)}$ and proved several structural lemmas
about~$A$.

\subsection{First-Target Lemma}\label{subsec:first-target}

\begin{proposition}[First-Target Lemma]
For every $k \notin A$, the smallest multiple
$m_k = k \cdot \ceil{(n+1)/k} \in (n, n+k]$ is not in $\NH(A)$
(i.e., no smooth element divides~$m_k$).
\end{proposition}

This was verified computationally with zero violations for
$n \leq 50{,}000$.  The proof uses the fact that for $d \in A$ to
divide~$m_k$, we would need $m_k / d \in (1, 3/(2c))$, but $c > 3/4$
for large~$n$ forces $3/(2c) < 2$, leaving no integer.

\subsection{Rough exclusion from tight sets}

\begin{proposition}
If $S^*$ maximizes the deficiency $\text{def}(S) = |S| - |\NH(S)|$,
then $S^*$ contains no element $k \in [cn, n]$ with $P^+(k) \geq 5$.
\end{proposition}

The proof uses a ``private target lemma'': removing such~$k$ from~$S^*$
decreases~$|S^*|$ by~$1$ but decreases~$|\NH(S^*)|$ by at least~$2$
(since $k$'s targets are exclusive from~$A$), increasing the deficiency
and contradicting maximality.

\emph{Consequence:} The K\"onig tight set consists entirely of smooth
numbers. This was confirmed computationally: at the critical
length $L = f(n) - 1$, the tight set is always 100\% inside~$A$.

\subsection{Shearer entropy bound}\label{subsec:shearer-detail}

The Shearer entropy bound was applied to the bipartite graph and found
to \emph{never fail}: at every tested~$n$ and every subset type, the
Shearer bound gives $|\NH(S)| > |S|$.  This contrasts sharply with
Cauchy--Schwarz, which fails for adversarial subsets.

The Shearer--CS dichotomy (Section~\ref{subsec:shearer-dichotomy}) was
first observed in the initial attempt and motivated the search for
alternatives to CS.

\subsection{Three remaining lemmas}

The initial proof attempt identified three lemmas needed to close
the gap:

\begin{enumerate}
\item \textbf{Restricted divisor count:} $\tau_S(h) \leq \tau(h; [h/2s, h/5])$
  (the number of divisors in a restricted range).  \emph{Status: Proved}
  (Lemma~2 of Section~\ref{sec:construction}).

\item \textbf{Geometric mean bound:} $\prod_{k \in S} \deg(k) \geq
  \prod_{k \in S} |S|^{1/|S|}$.  \emph{Status: Not needed} (subsumed
  by the CS framework).

\item \textbf{Shearer combination:} Combine the Shearer entropy bound
  with the restricted divisor count to get $|\NH(S)| \geq |S|$.
  \emph{Status: Attempted but not closed} (the analytic estimate of
  $H(\tau)$ is not tight enough).
\end{enumerate}

\subsection{Modular remainder and recursive doubling}

Two ``escape hatches'' were explored:

\begin{enumerate}
\item \textbf{Modular remainder:} If $k \in S$ and $k \equiv r \pmod{p}$
  for a small prime~$p$, then $k$'s multiples in~$H$ can be partitioned
  by residue class, potentially giving a private target.
  \emph{Status: Works for specific elements but does not generalize.}

\item \textbf{Recursive doubling:} Extend the doubling map
  $k \mapsto 2k$ beyond the top half, matching elements $k \in (n/4, n/2]$
  to $2k \in (n/2, n]$ if those targets are still available after the
  first round.  \emph{Status: Does not help because $2k$ may already
  be claimed.}
\end{enumerate}

\subsection{Summary}

The initial attempt made significant structural progress:
\begin{itemize}
\item The $U/V$ split and Case~A/B decomposition became the standard
  framework for all subsequent work.
\item The First-Target Lemma and rough exclusion narrow the search for
  tight sets to smooth numbers.
\item The Shearer bound's success (vs.\ CS's failure) highlighted that
  the problem is not one of ``not enough expansion'' but of ``expansion
  that is hard to certify.''
\item The computational verification of Hall's condition for small~$n$
  provided the foundation for the exhaustive HK campaign.
\end{itemize}

The core gap---proving Hall for Case~B with large subsets---remained
open and motivated the 43-approach investigation documented in this paper.

%% ===================================================================
\section{Conclusion: where the snakes lie}\label{sec:conclusion}
%% ===================================================================

\subsection{The fundamental obstacle}

The bipartite graph $G_n = (V, H, E)$ has a deceptively simple structure:
edges are given by divisibility, degrees are $M/k + O(1)$, and the
overall expansion ratio $|H|/|V| = M/N \to \infty$.  Yet proving
Hall's condition $|\NH(S)| \geq |S|$ for \emph{all} $S \subseteq V$
has resisted 43~analytic approaches.

The root cause is a single structural fact:

\begin{quote}
\emph{For vertices~$k$ near $N \approx n/2$, the degree
$d(k) \approx \de \approx 2$--$3$ is comparable to the maximum
codegree $D_2 \approx \de$.  Thus $D/D_2 \not\to \infty$.}
\end{quote}

This means:
\begin{itemize}
\item \textbf{Probabilistic methods fail} because they require
  $D / D_2 \to \infty$ (LLL, nibble, Janson all need degree to dominate
  codegree).
\item \textbf{Cauchy--Schwarz fails} because $E_2 \approx E_1^2 / |S|$
  (the codegree sum is comparable to the ``budget,'' leaving no margin).
\item \textbf{Spectral methods fail} because the graph is extremely
  irregular (degree varies from~$\de$ to~$M$), destroying spectral
  concentration.
\item \textbf{Sieve methods fail} because divisibility events are
  positively correlated, and the correlation structure is too complex
  for Bonferroni truncation.
\end{itemize}

\subsection{Why the gap is at exactly $2/\sqrt{e}$}

The current state of knowledge is:
\[
  \left(\frac{2}{\sqrt{e}} + o(1)\right) n\sqrt{\frac{\ln n}{\ln\ln n}}
  \;\leq\; f(n)
  \;\leq\; (1.7398\cdots + o(1))\, n\sqrt{\ln n}.
\]
The gap between lower and upper bounds is a factor of~$\sqrt{\ln\ln n}$.
The constant $C = 2/\sqrt{e}$ in the lower bound arises from a precise
balance.  The Erd\H{o}s--Pomerance argument shows that with
$L = (C - \eps) n \sqrt{\ln n / \ln\ln n}$, there are too few multiples
for the smooth numbers near~$N$ to be matched.  The parameter $\de = 2M/n - 1$ equals
$(2C - 2 + o(1))\sqrt{\ln n / \ln\ln n}$, and at $C = 2/\sqrt{e}$ this
gives $\de \sim 2(2/\sqrt{e} - 1)\sqrt{\ln n / \ln\ln n} \approx
0.426\sqrt{\ln n / \ln\ln n}$.

The growth rate $\sqrt{\ln n / \ln\ln n}$ is \emph{sublogarithmic}:
slower than any power of $\log n$, let alone polynomial.  This means
$\de(n)$ passes through the critical integer values $2, 3, 4, \ldots$
at enormous values of~$n$:

\begin{center}
\begin{tabular}{@{}cc@{}}
\toprule
$\de \geq$ & approximate $n$ \\
\midrule
2 & $3{,}600$ \\
3 & $2 \times 10^7$ \\
4 & $5 \times 10^{14}$ \\
5 & $\sim 10^{26}$ \\
\bottomrule
\end{tabular}
\end{center}

Any proof that requires ``$\de$ large'' (say $\de \geq 10$) would need
$n \gg 10^{100}$, far beyond computational verification.  The gap
between what can be verified ($n \leq 10^6$, where $\de < 3$) and what
can be proved asymptotically (requiring $\de \to \infty$) is a
\emph{desert} of moderate~$n$ where neither tool reaches.

\subsection{What a successful proof would need}

Based on our investigation, a proof of the upper bound~\eqref{eq:upper-bound}
would likely need one of:

\begin{enumerate}
\item \textbf{A new Hall certificate.}
  Some combinatorial or algebraic structure in $G_n$ that certifies
  $|\NH(S)| \geq |S|$ for all~$S$ simultaneously, without checking
  subsets individually.  The FMC approach comes closest but requires
  $\alpha(V_{\mathrm{rest}}) \geq 2$, which remains unproved.

\item \textbf{A new counting technique.}
  Something beyond Cauchy--Schwarz that controls the codegree sum
  $E_2$ more tightly.  The weighted CS (Section~\ref{subsec:weighted-cs})
  shows that the ``right'' weights exist and give margin~$\sim\!1.3$,
  but no analytic expression for them is known.

\item \textbf{A GCD graph argument.}
  The Koukoulopoulos--Maynard framework
  (Section~\ref{subsec:mult-energy}) is the most promising modern tool.
  The adversarial set is multiplicatively unstructured at large~$n$,
  which is the regime where the K--M expansion argument should work.
  Making this quantitative for the specific Erd\H{o}s~710 graph is the
  main open challenge.

\item \textbf{A topological or algebraic argument.}
  Hall's theorem is equivalent to the non-vanishing of a certain
  permanent.  Techniques from algebraic combinatorics (e.g., the
  Combinatorial Nullstellensatz) might certify this without subset
  enumeration.
\end{enumerate}

\subsection{Open questions}

\begin{enumerate}
\item Is $\alpha(V_{\mathrm{rest}}) \geq 2$ for all $n$ sufficiently large?
  (This would close the gap via the FMC theorem.)

\item Does the optimal weighted CS ratio $d^\top C^{-1} d / |S|$ remain
  bounded away from~$1$ as $n \to \infty$?  (It is $\sim\!1.3$ at all
  tested~$n$.)

\item Can the Koukoulopoulos--Maynard GCD graph technique be made
  quantitative enough to prove Hall for $G_n$?

\item Is there a ``forbidden subgraph'' characterization of the tight
  sets of~$G_n$?

\item Can the Shearer entropy bound be made analytic (i.e., can $H(\tau)$
  be bounded from below for all subsets)?
\end{enumerate}

\subsection{Summary table of approaches}

Table~\ref{tab:approaches} lists all 43~approaches investigated,
organized by category, with a one-line failure reason for each.

\begin{longtable}{@{}rlp{5.5cm}@{}}
\caption{Summary of 43 approaches to proving global Hall's condition.}
\label{tab:approaches} \\
\toprule
\# & Approach & Failure reason \\
\midrule
\endfirsthead
\toprule
\# & Approach & Failure reason \\
\midrule
\endhead
\midrule
\multicolumn{3}{r}{\footnotesize\emph{continued on next page}} \\
\endfoot
\bottomrule
\endlastfoot
\multicolumn{3}{l}{\textbf{Cauchy--Schwarz family}} \\
1 & Standard CS & Ratio 0.86 for adversarial $S$ \\
2 & Per-interval CS & Works locally; 87--93\% overlap kills global \\
3 & Weighted CS ($f = 1/\tau$) & Passes computationally; no analytic expression \\
4 & Optimal weighted CS ($C^{-1}d$) & Passes computationally; $C^{-1}$ intractable \\
5 & Filtered CS & Ratio 0.991 (threshold removes key targets) \\
6 & Truncated CS (Ford cap) & Ratio 0.41--0.49 \\
7 & Variable-constant CS & $C_{\mathrm{crit}} = 1.01$ stable, not $\to 1$ \\
\addlinespace
\multicolumn{3}{l}{\textbf{Matching \& fractional methods}} \\
8 & Greedy matching & 112/220 failures; locally optimal $\neq$ globally \\
9 & Uniform fractional & $\min w(k) = 0.06$; starves min-degree vertices \\
10 & Nonuniform fractional (LP) & Exists but doesn't constitute proof for all $n$ \\
11 & FMC with dyadic intervals & $\Sigma > 1$ for $n < 15$K (greedy) \\
12 & FMC three-block & Blocked on $\alpha(S_+)$ proof \\
13 & FMC $V_{\min}/V_{\mathrm{rest}}/S_-$ & Blocked on $\alpha(V_{\mathrm{rest}}) \geq 2$ \\
14 & Sinkhorn iteration & 13--15\% fractional deficiency \\
15 & CLP factoring & Graph not regular; truncation doesn't help \\
16 & $\sqrt{2}$-partition sequential & Local condition passes; global guarantee fails \\
\addlinespace
\multicolumn{3}{l}{\textbf{Probabilistic methods}} \\
17 & Symmetric LLL & $P \cdot e \cdot (D+1) \approx 10^4$ \\
18 & Target-centered LLL & Same; high-$\tau$ targets cause failure \\
19 & Janson inequality & $\Delta \gg (\sum P_i)^2$ \\
20 & Erd\H{o}s--Spencer weighted LLL & Circular obstruction in $x_i$ system \\
21 & Semi-random nibble (Z115) & 25--30\% of $|V|$ permanently unmatchable \\
\addlinespace
\multicolumn{3}{l}{\textbf{Sieve \& inclusion-exclusion}} \\
22 & Bonferroni (order 2) & Waste ratio 2.4--9.2$\times$, grows with $n$ \\
23 & Bonferroni (higher orders) & Signs diverge; no useful truncation \\
24 & Product-formula sieve & Overestimates $|\NH(S)|$ (wrong direction) \\
25 & Unique multiple sieve & 30--90\% of elements have $d_1 = 0$ \\
26 & Standard sieve (Selberg/Brun) & 3.56$\times$ safety factor, wrong direction \\
\addlinespace
\multicolumn{3}{l}{\textbf{Graph-theoretic \& spectral}} \\
27 & Spectral gap & Graph too irregular; $\sigma_2/\sigma_1$ large \\
28 & Haxell independent transversal & Overlap 87--93\% makes conflict graph dense \\
29 & Tur\'an bound & 27--32\% of target; gap grows with $n$ \\
30 & Degeneracy bound & 10--28\% of target; degeneracy $\sim 15\sqrt{n}$ \\
31 & Ford divisor cap & Ratio 0.41--0.49 \\
32 & Multiplicative energy (K--M) & Qualitative but not quantitative for this graph \\
\addlinespace
\multicolumn{3}{l}{\textbf{Partition \& structural}} \\
33 & Dyadic partition ($\times 2$) & 1--2 intervals fail degree condition \\
34 & $\sqrt{2}$-partition & All pass but no global guarantee \\
35 & Fine partition ($\times 1.1$) & All pass locally; same overlap problem \\
36 & Stratified Hall (octile) & 7/8 pass; Q0 fails until subdivided \\
37 & $V_{\min}$ disjointness & Works for $V_{\min}$; doesn't extend to $V_{\mathrm{rest}}$ \\
\addlinespace
\multicolumn{3}{l}{\textbf{Other}} \\
38 & Derandomization & No efficient random process to derandomize \\
39 & Modular remainder & Works for specific elements; doesn't generalize \\
40 & Recursive doubling & Targets already claimed \\
41 & Surplus-excess proof & Bonferroni waste kills it \\
42 & Neumann series bound & Codegree matrix not contractive \\
43 & Quasi-independence & Positive correlations in divisibility \\
\end{longtable}

\begin{figure}[ht]
\centering
\includegraphics[width=\textwidth]{fig_approach_summary.pdf}
\caption{Visual summary of all 43 approaches, organized by category.
  Red: dead (fundamental obstruction).  Orange: partial success
  (works locally or computationally).  Green: promising direction
  (not yet quantitative).}
\label{fig:summary}
\end{figure}

\subsection{Final words}

Erd\H{o}s Problem~\#710 has resisted a determined assault with every
tool in the modern combinatorialist's arsenal.  The computational
evidence is overwhelming: Hall's condition holds with zero failures
through $n = 10^6$, and no adversarial subset has ever been found with
expansion ratio below~$1$.  The gap between what we can compute and
what we can prove is a testament to the depth of the problem.

We hope that this detailed investigation---documenting not just what works
but, more importantly, what does \emph{not} work and \emph{why}---will
save future researchers from repeating these dead ends and guide them
toward the techniques most likely to succeed.


%% ============================================================
%% Bibliography
%% ============================================================
\begin{thebibliography}{99}

\bibitem{Erdos1938}
P.~Erd\H{o}s, On sequences of integers no one of which divides the product
of two others and on some related problems,
\emph{Mitt.\ Forsch.-Inst.\ Math.\ Mech.\ Univ.\ Tomsk} \textbf{2} (1938), 74--82.

\bibitem{ErdosPomerance1980}
P.~Erd\H{o}s and C.~Pomerance,
Matching the natural numbers up to~$n$ with distinct multiples in another interval,
\emph{Nederl.\ Akad.\ Wetensch.\ Proc.\ Ser.~A} \textbf{83} (1980), 147--161.

\bibitem{ErdosSarkozy1966}
P.~Erd\H{o}s, A.~S\'ark\"ozy, and E.~Szemer\'edi,
On divisibility properties of sequences of integers,
\emph{Studia Sci.\ Math.\ Hungar.} \textbf{1} (1966), 431--435.

\bibitem{Hall1935}
P.~Hall, On representatives of subsets,
\emph{J.\ London Math.\ Soc.} \textbf{10} (1935), 26--30.

\bibitem{HopcroftKarp1973}
J.\,E.~Hopcroft and R.\,M.~Karp,
An $n^{5/2}$ algorithm for maximum matchings in bipartite graphs,
\emph{SIAM J.\ Comput.} \textbf{2}(4) (1973), 225--231.

\bibitem{Ford2008}
K.~Ford, The distribution of integers with a divisor in a given interval,
\emph{Ann.\ of Math.} (2) \textbf{168} (2008), 367--433.

\bibitem{KoukoulopoulosMaynard2020}
D.~Koukoulopoulos and J.~Maynard,
On the Duffin--Schaeffer conjecture,
\emph{Ann.\ of Math.} (2) \textbf{192} (2020), 251--307.

\bibitem{HildebrandTenenbaum1993}
A.~Hildebrand and G.~Tenenbaum,
Integers without large prime factors,
\emph{J.\ Th\'eor.\ Nombres Bordeaux} \textbf{5} (1993), 411--484.

\bibitem{Rodl1985}
V.~R\"odl, On a packing and covering problem,
\emph{European J.\ Combin.} \textbf{6} (1985), 69--78.

\bibitem{MolloyReed2002}
M.~Molloy and B.~Reed,
\emph{Graph Colouring and the Probabilistic Method},
Springer, Algorithms and Combinatorics~23, 2002.

\bibitem{Haxell1995}
P.\,E.~Haxell, A condition for matchability in hypergraphs,
\emph{Graphs Combin.} \textbf{11} (1995), 245--248.

\bibitem{ErdosSpencer1991}
P.~Erd\H{o}s and J.~Spencer,
Lopsided Lov\'asz Local Lemma and Latin transversals,
\emph{Discrete Appl.\ Math.} \textbf{30} (1991), 151--154.

\bibitem{Janson1998}
S.~Janson, New versions of Suen's correlation inequality,
\emph{Random Structures Algorithms} \textbf{13} (1998), 467--483.

\bibitem{Erdos1990}
P.~Erd\H{o}s, Some of my favourite unsolved problems,
in \emph{A Tribute to Paul Erd\H{o}s} (A.~Baker, B.~Bollob\'as,
A.~Hajnal, eds.), Cambridge University Press, 1990, pp.~467--478.

\bibitem{Guy2004}
R.\,K.~Guy, \emph{Unsolved Problems in Number Theory}, 3rd~ed.,
Springer, 2004.

\end{thebibliography}

\end{document}
