%% ===================================================================
\section{Results from the initial proof attempt}\label{sec:initial}
%% ===================================================================

Before the systematic investigation documented in
Sections~\ref{sec:cs-family}--\ref{sec:graph-theoretic}, an initial
proof attempt (Sessions~1--9) established several structural results
and identified the core difficulty.  We summarize these here both for
completeness and because some ideas remain promising.

\subsection{Proof reduction: $U/V$ split and Case~A/B}

The initial attempt established the same proof architecture described
in Sections~\ref{sec:construction}--\ref{subsec:cs-framework}:
\begin{enumerate}
\item Top half~$U$ matched by doubling ($k \mapsto 2k$).
\item Bottom half~$V$ requires Hall's condition in $G_n = (V, H, E)$.
\item Case~A ($\min(S) \leq M/(s+1)$) handled by single-element degree.
\item Case~B ($\min(S) > M/(s+1)$) is the remaining gap.
\end{enumerate}

Additionally, the initial attempt introduced a finer decomposition
using a ``smooth set'' $A = \{k \in [cn, n] : P^+(k) \leq B\}$
with $c = e^{-1/(2u)}$ and proved several structural lemmas
about~$A$.

\subsection{First-Target Lemma}\label{subsec:first-target}

\begin{proposition}[First-Target Lemma]
For every $k \notin A$, the smallest multiple
$m_k = k \cdot \ceil{(n+1)/k} \in (n, n+k]$ is not in $\NH(A)$
(i.e., no smooth element divides~$m_k$).
\end{proposition}

This was verified computationally with zero violations for
$n \leq 50{,}000$.  The proof uses the fact that for $d \in A$ to
divide~$m_k$, we would need $m_k / d \in (1, 3/(2c))$, but $c > 3/4$
for large~$n$ forces $3/(2c) < 2$, leaving no integer.

\subsection{Rough exclusion from tight sets}

\begin{proposition}
If $S^*$ maximizes the deficiency $\text{def}(S) = |S| - |\NH(S)|$,
then $S^*$ contains no element $k \in [cn, n]$ with $P^+(k) \geq 5$.
\end{proposition}

The proof uses a ``private target lemma'': removing such~$k$ from~$S^*$
decreases~$|S^*|$ by~$1$ but decreases~$|\NH(S^*)|$ by at least~$2$
(since $k$'s targets are exclusive from~$A$), increasing the deficiency
and contradicting maximality.

\emph{Consequence:} The K\"onig tight set consists entirely of smooth
numbers. This was confirmed computationally: at the critical
length $L = f(n) - 1$, the tight set is always 100\% inside~$A$.

\subsection{Shearer entropy bound}\label{subsec:shearer-detail}

The Shearer entropy bound was applied to the bipartite graph and found
to \emph{never fail}: at every tested~$n$ and every subset type, the
Shearer bound gives $|\NH(S)| > |S|$.  This contrasts sharply with
Cauchy--Schwarz, which fails for adversarial subsets.

The Shearer--CS dichotomy (Section~\ref{subsec:shearer-dichotomy}) was
first observed in the initial attempt and motivated the search for
alternatives to CS.

\subsection{Three remaining lemmas}

The initial proof attempt identified three lemmas needed to close
the gap:

\begin{enumerate}
\item \textbf{Restricted divisor count:} $\tau_S(h) \leq \tau(h; [h/2s, h/5])$
  (the number of divisors in a restricted range).  \emph{Status: Proved}
  (Lemma~2 of Section~\ref{sec:construction}).

\item \textbf{Geometric mean bound:} $\prod_{k \in S} \deg(k) \geq
  \prod_{k \in S} |S|^{1/|S|}$.  \emph{Status: Not needed} (subsumed
  by the CS framework).

\item \textbf{Shearer combination:} Combine the Shearer entropy bound
  with the restricted divisor count to get $|\NH(S)| \geq |S|$.
  \emph{Status: Attempted but not closed} (the analytic estimate of
  $H(\tau)$ is not tight enough).
\end{enumerate}

\subsection{Modular remainder and recursive doubling}

Two ``escape hatches'' were explored:

\begin{enumerate}
\item \textbf{Modular remainder:} If $k \in S$ and $k \equiv r \pmod{p}$
  for a small prime~$p$, then $k$'s multiples in~$H$ can be partitioned
  by residue class, potentially giving a private target.
  \emph{Status: Works for specific elements but does not generalize.}

\item \textbf{Recursive doubling:} Extend the doubling map
  $k \mapsto 2k$ beyond the top half, matching elements $k \in (n/4, n/2]$
  to $2k \in (n/2, n]$ if those targets are still available after the
  first round.  \emph{Status: Does not help because $2k$ may already
  be claimed.}
\end{enumerate}

\subsection{Summary}

The initial attempt made significant structural progress:
\begin{itemize}
\item The $U/V$ split and Case~A/B decomposition became the standard
  framework for all subsequent work.
\item The First-Target Lemma and rough exclusion narrow the search for
  tight sets to smooth numbers.
\item The Shearer bound's success (vs.\ CS's failure) highlighted that
  the problem is not one of ``not enough expansion'' but of ``expansion
  that is hard to certify.''
\item The computational verification of Hall's condition for small~$n$
  provided the foundation for the exhaustive HK campaign.
\end{itemize}

The core gap---proving Hall for Case~B with large subsets---remained
open and motivated the 43-approach investigation documented in this paper.
