%% ===================================================================
\section{Interesting phenomena}\label{sec:phenomena}
%% ===================================================================

Beyond the proof attempts, our computational investigation uncovered
several striking phenomena that illuminate the structure of the
Erd\H{o}s~710 bipartite graph.

\subsection{Sawtooth oscillations}\label{subsec:sawtooth}

\begin{observation}[Z65]\label{obs:sawtooth}
The FMC sum $\Sigma(n) = \sum_{j=1}^{J(n)} 1/\alpha_j$ exhibits a
\emph{sawtooth} pattern as a function of~$n$, with sharp drops at values
of~$n$ where the number of dyadic intervals $J(n)$ increases by~$1$.
\end{observation}

This is perhaps the most visually striking phenomenon in the entire
investigation, and it reveals the mechanism by which the problem
``breathes'' as~$n$ grows.

\paragraph{The mechanism.}
Within a fixed $J$-regime (fixed number of dyadic intervals), the FMC
sum rises steadily: as~$n$ increases, new smooth numbers enter the
bottom interval, the codegree accumulates, and the expansion ratio
$\alpha_j$ of the bottleneck interval slowly deteriorates.  Then,
at a critical value of~$n$, the number of intervals jumps
$J \to J + 1$: the hardest interval gets split in two, each with
better degree homogeneity.  The FMC sum drops sharply---a
``shockwave''---and the cycle begins again.

The transitions are non-monotonic.  At some values of~$n$, $J$
temporarily \emph{decreases} (an interval becomes empty as the
smoothness bound~$B$ shifts), causing an upward shock:

\begin{center}
\begin{tabular}{@{}ccl@{}}
\toprule
$n$ & transition & direction \\
\midrule
140 & $J\!: 2 \to 3$ & $\uparrow$ new interval \\
280 & $J\!: 4 \to 3$ & $\downarrow$ interval empties \\
520 & $J\!: 3 \to 4$ & $\uparrow$ \\
2{,}050 & $J\!: 4 \to 5$ & $\uparrow$ \\
8{,}200 & $J\!: 5 \to 6$ & $\uparrow$ \\
34{,}500 & $J\!: 6 \to 7$ & $\uparrow$ \\
68{,}500 & $J\!: 7 \to 8$ & $\uparrow$ \\
266{,}000 & $J\!: 8 \to 9$ & $\uparrow$ \\
\bottomrule
\end{tabular}
\end{center}

\paragraph{The peak envelope.}
The peak of each $J$-regime defines an envelope tracking the
worst-case FMC sum:

\begin{center}
\begin{tabular}{@{}rcc@{}}
\toprule
$J$ & peak $n$ & peak $\Sigma$ \\
\midrule
4 & 520 & 0.793 \\
5 & 2{,}050 & 0.850 \\
6 & 8{,}200 & 0.831 \\
7 & 34{,}500 & 0.857 \\
8 & 68{,}500 & \textbf{0.881} \\
9 & 266{,}000 & 0.871 \\
\bottomrule
\end{tabular}
\end{center}

The global maximum over all 461~data points is $\Sigma = 0.881$
at $n = 68{,}500$ (within the $J = 8$ regime), leaving an $11.9\%$
margin to the threshold~$1$.  The envelope is \emph{non-monotonic}:
$J = 6$ has a lower peak than $J = 5$, and $J = 9$ lower than $J = 8$.
This irregular growth makes extrapolation to large~$n$ unreliable---we
cannot determine whether the envelope eventually reaches~$1$ or
stabilizes below it.

\paragraph{The companion curves.}
Figure~\ref{fig:sawtooth} shows two curves:
$\sum 1/\CS_{\mathrm{ref},j}$ (the CS-based upper bound, blue) and
$\sum 1/\alpha_j$ (the greedy expansion ratio, green).  The greedy
curve tracks below the CS curve, confirming that the true expansion
is better than CS predicts.  The bottom panel shows $\delta(n)$
(orange, growing sublogarithmically) and $J(n)$ (purple, step function),
revealing the $J$-transitions that drive the sawtooth.

\begin{figure}[ht]
\centering
\includegraphics[width=\textwidth]{fig_sawtooth.pdf}
\caption{The FMC sawtooth.  \textbf{Top:} FMC sum
  $\sum_j 1/\alpha_j$ (green) and $\sum_j 1/\CS_{\mathrm{ref},j}$
  (blue) vs.~$n$ on a log scale, from 461~data points.
  Sharp drops (``shockwaves'') occur at $J$-transitions where the
  number of dyadic intervals changes.  The red dashed envelope tracks
  the peaks, reaching a global max of $0.881$ at $n = 68{,}500$.
  The green shaded region shows the $11.9\%$ margin to the
  threshold~$1$.
  \textbf{Bottom:} The minimum-degree parameter $\delta(n)$ (orange)
  and the number of intervals $J(n)$ (purple step function).}
\label{fig:sawtooth}
\end{figure}

\subsection{CS deficiency oscillation}\label{subsec:cs-oscillation}

\begin{observation}[Z113b, Z114]\label{obs:cs-deficiency}
The global Cauchy--Schwarz ratio $\CS(T_0)$ for the adversarial subset
oscillates near~$1.0$ as $n$ increases, without converging in either
direction.
\end{observation}

At $C_{\mathrm{mult}} = 1.00$ (the base constant), $\CS(T_0)$ crosses
above~$1$ at some~$n$ values and below~$1$ at others.  The oscillation
appears to be driven by the same sawtooth mechanism: the adversarial
subset~$T_0$ restructures when the number of dyadic intervals changes.

This non-monotonic behavior is one reason the variable-constant approach
(Section~\ref{subsec:variable-cs}) fails: there is no smooth function
$C_{\mathrm{crit}}(n) \to 1$ to track.

\subsection{GCD stratum decomposition}\label{subsec:gcd-stratum}

\begin{observation}\label{obs:gcd-stratum}
The codegree sum (``off-diagonal'' part of~$E_2$) is dominated by
pairs with moderate GCD: the stratum $\gcd(k_1, k_2) \in [11, 500]$
accounts for $80$--$95\%$ of the truncated codegree sum, while coprime
pairs ($\gcd = 1$) contribute only $0$--$2\%$.
\end{observation}

The concentration in moderate GCDs reflects the ``anatomy of smooth
numbers'': elements of the adversarial set tend to share $2$--$3$ small
prime factors (giving $\gcd \in [6, 500]$), which is enough to create
codegree without making the lcm exceed the truncation threshold.

Interestingly, the lcm truncation (excluding pairs with
$\lcm(k_1, k_2) > n + L$) removes only $3$--$11\%$ of the codegree sum.
The truncation is nearly invisible above $\gcd = 10$.

\subsection{Far-partner stability}\label{subsec:far-partner}

\begin{observation}\label{obs:far-partner}
The ``far-partner fraction''---the fraction of vertices~$k$ whose
codegree partner~$k'$ with $\lcm(k, k') > n + L$ contributes to
the harmonic sum---stabilizes at $8$--$13\%$ and does \emph{not}
decay to~$0$ with~$n$.
\end{observation}

This means that $\sim\!10\%$ of all vertices are ``forced far'' by the
product constraint, creating a persistent structural contribution to the
codegree.  The far-partner mechanism is driven by coprime elements
($\gcd(k, k') = 1$) at opposite ends of the interval.

\subsection{Shearer vs.\ CS dichotomy}\label{subsec:shearer-dichotomy}

\begin{observation}[Initial attempt]\label{obs:shearer}
Shearer's entropy method~\cite{ErdosSpencer1991} \emph{never fails} on
the Erd\H{o}s~710 graph: the Shearer bound gives $|\NH(S)| \geq C |S|$
with $C > 1$ at every tested~$n$ and every subset type.  Meanwhile,
Cauchy--Schwarz fails on adversarial subsets.
\end{observation}

The Shearer bound used here is
$|\NH(S)| \geq \bigl(\prod_{k \in S} \deg(k)\bigr)^{1/\Delta}$,
where $\Delta = \max_{h \in \NH(S)} \tau_S(h)$ is the maximum
right-degree.  Taking logarithms, this becomes
$\log|\NH(S)| \geq (1/\Delta) \sum_{k \in S} \log \deg(k)$.
This bound is stronger than CS when the multiplicity distribution
is concentrated (many targets with $\tau = 1$--$2$), which is
exactly the case for our graph.

However, Shearer's bound cannot be made into a proof either: it requires
\emph{computing} $H(\tau)$ for specific subsets, and no analytic estimate
of $H(\tau)$ is tight enough.

\subsection{LCM independent set margin}\label{subsec:lcm-margin}

\begin{observation}[Z09]\label{obs:lcm-margin}
The LCM independent set (vertices with pairwise $\lcm > n + L$) always
covers at least $|V|$ targets, with the tightest margin being
$|\NH(I_{\mathrm{LCM}})| / |I_{\mathrm{LCM}}| = 1.012$.
\end{observation}

This means that even after restricting to a maximal set of
``non-overlapping'' vertices, the target count barely exceeds the
vertex count.  The $1.2\%$ margin underscores the razor-thin expansion.
