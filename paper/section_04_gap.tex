%% ===================================================================
\section{The critical gap}\label{sec:gap}
%% ===================================================================

The central negative result of our investigation is that no analytic
argument we have found can bridge the gap between per-interval Hall's
condition (proved in Section~\ref{subsec:per-interval-cs}) and global
Hall's condition.  This section characterizes the gap precisely.

\subsection{Per-interval Hall does not imply global Hall}\label{subsec:gap-statement}

Let $I_1, \ldots, I_J$ be the dyadic intervals partitioning~$V$.
Per-interval Hall states that $|\NH(S \cap I_j)| \geq |S \cap I_j|$
for each~$j$.  The naive attempt to globalize is:
\begin{align*}
  |\NH(S)| &= \Bigl|\bigcup_j \NH(S \cap I_j)\Bigr| \\
  &\geq \sum_j |\NH(S \cap I_j)| - \text{overlaps}
  \;\geq\; |S| - \text{overlaps}.
\end{align*}
This works only if the overlap is zero.  In practice, the overlap is enormous.

\subsection{Cross-interval overlap}\label{subsec:overlap}

\begin{observation}[Z29a]\label{obs:overlap}
Adjacent dyadic intervals share $87$--$93\%$ of their targets.
Non-adjacent intervals share similarly.  At $n = 50{,}000$, the total
pairwise overlap exceeds the total surplus by a factor of~$3.3$.
\end{observation}

The data from Z29:

\begin{center}
\begin{tabular}{@{}rrrrrr@{}}
\toprule
$n$ & intervals & $|V|$ & surplus & overlap & margin \\
\midrule
1{,}000 & 4 & 268 & 619 & 998 & $-379$ \\
5{,}000 & 5 & 1{,}373 & 5{,}059 & 11{,}269 & $-6{,}211$ \\
10{,}000 & 6 & 2{,}695 & 12{,}296 & 32{,}243 & $-19{,}947$ \\
50{,}000 & 7 & 13{,}001 & 76{,}395 & 253{,}043 & $-176{,}648$ \\
\bottomrule
\end{tabular}
\end{center}

Here ``surplus'' is $\sum_j (|\NH(I_j)| - |I_j|)$, ``overlap'' is
$\sum_{j < j'} |\NH(I_j) \cap \NH(I_{j'})|$, and ``margin'' is
surplus~$-$~overlap.  The margin is \emph{deeply negative} and worsening
with~$n$.

\subsection{Target multiplicity}\label{subsec:multiplicity}

\begin{observation}[Z29b]\label{obs:multiplicity}
Most targets are shared by multiple intervals.  At $n = 50{,}000$:
$\mu_{\max} = 7$ (equal to the number of intervals), $\mu_{\mathrm{avg}}
= 4.74$, and only $5.3\%$ of targets are unique to one interval.
\end{observation}

A ``max-multiplicity'' bridge argument would require
$\min_j \alpha_j \geq \mu_{\max} \approx \frac{1}{4}\log_2 n$.
Since $\min_j \alpha_j \approx 1.7$--$2.7$ at tested values of~$n$,
this approach fails by an order of magnitude.

\subsection{Global Cauchy--Schwarz failure}\label{subsec:global-cs}

\begin{observation}[Z112g, Z113b, Z114]\label{obs:global-cs}
The global Cauchy--Schwarz ratio $\CS(S) = E_1^2 / (|S| \cdot E_2)$ falls
below~$1$ for adversarial subsets~$S$ at all tested $n \geq 10{,}000$.
\end{observation}

The data from Z114 at the base constant $C = 2/\sqrt{e}$ (with
$\eps = 0.05$, $C_{\mathrm{mult}} = 1.00$):

\begin{center}
\begin{tabular}{@{}rcc@{}}
\toprule
$n$ & $\CS(T_0)$ & $|T_0|/|V|$ \\
\midrule
2{,}000 & 1.027 & 45.4\% \\
5{,}000 & 1.032 & 47.3\% \\
7{,}000 & 1.005 & 48.1\% \\
10{,}000 & \textbf{0.991} & 48.9\% \\
20{,}000 & \textbf{0.992} & 50.3\% \\
50{,}000 & \textbf{0.998} & 52.0\% \\
100{,}000 & \textbf{0.988} & 53.1\% \\
\bottomrule
\end{tabular}
\end{center}

Bold entries are below~$1$: the Cauchy--Schwarz bound \emph{does not}
prove Hall's condition for these subsets.  The values oscillate around~$1.0$
without converging in either direction (see Section~\ref{sec:phenomena}
on the oscillation phenomenon).

\begin{figure}[ht]
\centering
\includegraphics[width=\textwidth]{fig_global_cs_oscillation.pdf}
\caption{Top: global Cauchy--Schwarz ratio $\CS(T_0)$ for the
  adversarial subset at the base constant, oscillating around the
  threshold~$1$.  Bottom: the fraction $|T_0|/|V|$ of vertices in the
  adversarial subset, growing from $45\%$ to $53\%$.}
\label{fig:cs-oscillation}
\end{figure}

\subsection{The K\"onig deficient set}\label{subsec:konig}

The adversarial subset~$T_0$ achieving the minimum CS ratio has a
distinctive structure:

\begin{observation}\label{obs:konig}
The greedy-adversarial subset~$T_0$ spans approximately $48$--$53\%$ of
all vertices in~$V$, drawn from \emph{all} dyadic intervals at roughly
$60\%$ density per interval, and from \emph{all} degree classes.
\end{observation}

This is the core difficulty: the worst-case subset is not localized to
a single interval or degree band---it is a diffuse, structured set that
exploits the entire graph.

\subsection{The expansion ratio}\label{subsec:expansion}

\begin{observation}[Z112n]\label{obs:expansion}
The greedy-adversarial lower bound on the expansion ratio
$\alpha(V) = \min_{\emptyset \neq S \subseteq V} |\NH(S)| / |S|$
lies in the interval $(1.00, 1.25)$ for all tested $n \geq 3{,}000$.
The true minimum may be lower (the greedy heuristic does not
find the worst-case subset), but the exhaustive HK verification
confirms $\alpha(V) \geq 1$ for all $n \leq 10^6$.
\end{observation}

This razor-thin margin---expansion barely exceeding~$1$---is why every
approach fails.  A proof must establish $\alpha(V) \geq 1$ for all~$n$,
but the margin leaves no room for the $O(1)$ errors inherent in
asymptotic arguments.

\subsection{Why the gap persists}\label{subsec:why-gap}

The situation can be summarized as a trilemma:

\begin{enumerate}
\item \textbf{Per-interval analysis is too local.}
  Each interval sees strong expansion ($\CS \to \infty$), but combining
  intervals destroys the guarantee because neighborhoods overlap massively.

\item \textbf{Global analysis is too weak.}
  The global Cauchy--Schwarz bound oscillates around~$1.0$ for adversarial
  subsets, dipping to $\sim\!0.988$ at some values of~$n$---tantalizingly
  close to~$1$ but on the wrong side.

\item \textbf{Partition-based methods need $\alpha(V_{\mathrm{rest}}) \geq 2$.}
  The FMC theorem reduces the problem to bounding the expansion ratio
  of each block.  The $V_{\min}$ block is handled, but no analytic argument
  proves $\alpha(V_{\mathrm{rest}}) \geq 2$, even though the greedy
  heuristic gives $\alpha(V_{\mathrm{rest}}) \geq 2.3$ at all tested~$n$.
\end{enumerate}

The remainder of this paper catalogs 43~approaches that attempt to
resolve this trilemma, organized by technique.
