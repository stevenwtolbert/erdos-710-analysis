%% ===================================================================
\section{Sieve and inclusion-exclusion methods}\label{sec:sieve}
%% ===================================================================

Sieve methods estimate $|\NH(S)|$ by subtracting the number of targets
\emph{not} hit by any element of~$S$.  Inclusion-exclusion gives an
exact formula, but its partial sums (Bonferroni bounds) diverge for
our graph.

\subsection{Bonferroni bounds (all orders)}\label{subsec:bonferroni}

\paragraph{Method.}
The exact neighborhood size is
\[
  |\NH(S)| = \sum_{h \in H} \mathbf{1}[\tau_S(h) \geq 1]
  = \sum_{t=1}^{|S|} (-1)^{t+1} \binom{|S|}{t}^{-1}
    \sum_{T \subseteq S,\, |T|=t} |\NH(T)|.
\]
The first-order (union) bound gives $|\NH(S)| \leq E_1$.  The
second-order Bonferroni bound subtracts pairwise overlaps:
\[
  |\NH(S)| \geq E_1 - \textstyle\sum_{k < k'} \deg(k, k')
  = E_1 - \text{codegree sum}.
\]
For this to prove Hall ($|\NH(S)| \geq s$), we need the codegree sum
$\leq E_1 - s = \text{``excess.''}$

\paragraph{Results (Z07).}
The Bonferroni bound fails in \emph{all} 80~tested configurations at
\emph{all} values of~$n$.  The ``waste ratio'' (codegree sum / true overlap)
ranges from $2.4$ to $9.2$ and \emph{grows} with~$n$:

\begin{center}
\begin{tabular}{@{}rcc@{}}
\toprule
$n$ & min waste ratio & max waste ratio \\
\midrule
500 & 2.39 & 5.72 \\
1{,}000 & 2.32 & 6.61 \\
2{,}000 & 2.39 & 7.69 \\
3{,}000 & 3.02 & 8.36 \\
5{,}000 & 3.11 & 9.22 \\
\bottomrule
\end{tabular}
\end{center}

\paragraph{Root cause.}
The codegree sum counts \emph{pairs} sharing a target: if a target~$h$
has $\tau = \tau_S(h)$ divisors from~$S$, the codegree sum gets
$\binom{\tau}{2}$, but the true overlap is only $\tau - 1$.  The ratio
$\binom{\tau}{2} / (\tau - 1) = \tau / 2$ grows with~$\tau$.  Since
many targets have $\tau \geq 5$--$30$, the Bonferroni bound is
quadratically wasteful.

Higher-order Bonferroni terms ($t = 3, 4, \ldots$) alternate in sign
and grow even faster, so the series does not truncate usefully.

\paragraph{Verdict: DEAD.}

\subsection{Product-formula sieve}\label{subsec:product-sieve}

\paragraph{Method.}
The independent sieve estimate: the number of targets hit by \emph{no}
element of~$S$ is approximately
$\text{sifted} \approx M \cdot \prod_{k \in S}(1 - 1/k) \approx M
e^{-\sigma}$, where $\sigma = \sum_{k \in S} 1/k$.  Then
$|\NH(S)| \approx M - M e^{-\sigma} = M(1 - e^{-\sigma})$.

\paragraph{Results (Z08).}
The actual sifted count is \emph{always larger} than the independent
estimate (ratio $2$--$300\times$).  The divisibility events $k_1 \mid h$
and $k_2 \mid h$ are \emph{positively correlated} when $\gcd(k_1, k_2)
> 1$, so more targets survive sifting than independence predicts.
The product formula \emph{overestimates} $|\NH(S)|$ and bounds in the
wrong direction.

\paragraph{Verdict: DEAD.}  Gives upper, not lower, bound.

\subsection{Unique multiple sieve}\label{subsec:unique-sieve}

\paragraph{Method.}
Show every $k \in S$ has at least one \emph{unique} multiple
$h \in H$ (with $\tau_S(h) = 1$).  This immediately gives
$|\NH(S)| \geq |S|$.

\paragraph{Results.}
Even for top-packed subsets, elements with $d_1(k) = 0$ (no unique
multiples) appear at $s/N \geq 0.3$.  For adversarial subsets,
$30$--$90\%$ of elements have no unique multiples.  The elements with
$d_1 = 0$ are large~$k$ near~$N$ whose few multiples are all shared.

\paragraph{Verdict: DEAD} for general subsets.

\subsection{Standard sieve}\label{subsec:standard-sieve}

\paragraph{Method.}
Apply the Selberg sieve or Brun sieve to estimate the number of
integers in~$H$ not divisible by any $k \in S$.

\paragraph{Results.}
The standard sieve gives a safety factor of $\sim\!3.56$ (the sifted
count is $3.56\times$ larger than the independent estimate), but in the
wrong direction: it \emph{overestimates} the sifted count, meaning it
\emph{underestimates} $|\NH(S)|$.  The sieve's systematic error comes
from the same positive correlations noted above.

\paragraph{Verdict: DEAD.}
