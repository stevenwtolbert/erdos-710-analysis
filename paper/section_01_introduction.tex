%% ===================================================================
\section{Introduction}\label{sec:intro}
%% ===================================================================

\subsection{The problem}

For a positive integer~$n$, define $f(n)$ to be the minimum $L \in \mathbb{Z}_{>0}$
such that there exists an injection
$\varphi\colon \{1,\ldots,n\} \to (n,\, n+L] \cap \mathbb{Z}$
with $k \mid \varphi(k)$ for all~$k$.
Thus $f(n)$ is the length of the shortest interval $(n, n+L]$ admitting
a \emph{divisible injection} from~$\{1,\ldots,n\}$.
This function was studied by Erd\H{o}s and
Pomerance~\cite{ErdosPomerance1980}, and its asymptotic determination
appears as Problem~\#710 on the Erd\H{o}s problems
database~\cite{Erdos1990,Guy2004}.
The requirement $\varphi(k) > n$ ensures that $\varphi(k) \neq k$, and
$k \mid \varphi(k)$ constrains each image to lie among the multiples of~$k$
that exceed~$n$.

The problem sits at the intersection of multiplicative number theory, extremal
combinatorics, and the anatomy of integers. It asks: how densely can we pack
the integers $1,\ldots,n$ into a short interval above~$n$, with each integer
assigned to one of its own multiples?

\subsection{History and prior work}\label{subsec:history}

The problem has its roots in Erd\H{o}s's 1938 Tomsk paper~\cite{Erdos1938}, which
introduced \emph{$2$-primitive sets}---sets where no element divides the product
of two distinct others---and established foundational density estimates.
The Erd\H{o}s--S\'ark\"ozy--Szemer\'edi collaboration~\cite{ErdosSarkozy1966}
produced a series of at least ten papers on divisibility properties of integer
sequences through the 1960s--70s, culminating in precise characterizations
of extremal sets.

The specific function~$f(n)$ and its asymptotic behavior were studied by
Erd\H{o}s and Pomerance~\cite{ErdosPomerance1980}, who established the lower bound
\begin{equation}\label{eq:lower-bound}
  f(n) \;\geq\; \left(\frac{2}{\sqrt{e}} - o(1)\right)
  n\sqrt{\frac{\ln n}{\ln\ln n}}.
\end{equation}
Their argument counts the available multiples of~$k$ in short intervals and shows
that no interval shorter than the right-hand side
of~\eqref{eq:lower-bound} can accommodate a divisible injection.
The constant $2/\sqrt{e} \approx 1.2131$ arises from optimizing over the
``smooth number'' parameter: elements~$k$ whose largest prime factor
$P(k) \leq k^{1/u}$ (with $u = \sqrt{\ln n / \ln\ln n}$) are the bottleneck,
and the Dickman function~$\rho(u)$ governs their density.

Erd\H{o}s and Pomerance also established the best known upper bound:
\begin{equation}\label{eq:known-upper}
  f(n) \;\leq\; (1.7398\cdots + o(1))\, n\sqrt{\ln n}.
\end{equation}
Note the absence of the $\ln\ln n$ correction: the upper
bound~\eqref{eq:known-upper} exceeds the conjectured
answer~\eqref{eq:lower-bound} by a factor of~$\sqrt{\ln\ln n}$,
which diverges (albeit slowly).  Closing this gap---proving the matching
upper bound
\begin{equation}\label{eq:upper-bound}
  f(n) \;\leq\; \left(\frac{2}{\sqrt{e}} + o(1)\right)
  n\sqrt{\frac{\ln n}{\ln\ln n}}
\end{equation}
and thereby determining $f(n)$ asymptotically---has remained open
since~1980. The problem appears in Guy's \emph{Unsolved Problems in
Number Theory}~\cite{Guy2004} (Sections~C9 and~C10) and carries
a \$500 prize on \texttt{erdosproblems.com}.

\subsection{Our contribution}

This paper does not prove~\eqref{eq:upper-bound}. Instead, we report on an
extensive investigation---both computational and analytic---that:
\begin{enumerate}[label=(\alph*)]
\item \textbf{Verifies computationally} that the upper bound holds for
  all $n \leq 10^6$, via exhaustive Hopcroft--Karp maximum bipartite matching
  with zero failures (Section~\ref{sec:proved});
\item \textbf{Proves partial results:} per-interval Hall's condition via
  Cauchy--Schwarz, the $V_{\min}$ disjointness lemma, the FMC theorem, and
  the small-$s$ regime (Section~\ref{sec:proved});
\item \textbf{Documents 43~distinct analytic approaches} to closing the gap
  between per-interval and global Hall's condition, each of which fails for
  identifiable structural reasons
  (Sections~\ref{sec:cs-family}--\ref{sec:graph-theoretic});
\item \textbf{Identifies the precise obstruction:} the worst-case
  subset~$T^*$ spans~$\sim\!48$--$53\%$ of all vertices, drawn from
  every dyadic interval at $\sim\!60\%$ density, with the global
  expansion ratio $\alpha(V) = \min_{S} |\NH(S)|/|S|$ barely
  exceeding~$1$ (Section~\ref{sec:gap});
\item \textbf{Catalogs interesting computational phenomena}---sawtooth
  oscillations, phase transitions, the Shearer--CS dichotomy---that may
  guide future work (Section~\ref{sec:phenomena}).
\end{enumerate}

The spirit of this paper is that of a detailed postmortem
of a prolonged research effort, written so that future researchers need not
repeat the same dead ends. We believe the negative results are themselves
valuable, both for the structural insight they provide and for the sharp
quantitative bounds on where each approach breaks.

\subsection{Why the problem is hard}\label{subsec:why-hard}

The fundamental difficulty can be stated in a single sentence:

\begin{quote}
\emph{The bipartite graph $G_n = (V, H, E)$ has per-interval expansion
$\gg 1$ but global expansion ratio barely exceeding~$1$, and no
known analytic method can certify that the expansion ratio remains
above~$1$ as $n \to \infty$.}
\end{quote}

More precisely:
\begin{itemize}
\item Within each dyadic interval $I_j = \{k \in V : 2^{j-1} < k \leq 2^j\}$,
  the Cauchy--Schwarz bound proves $|\NH(S \cap I_j)| \geq C_j |S \cap I_j|$
  with $C_j \to \infty$ as $n \to \infty$.
\item But the neighborhoods $\NH(S \cap I_j)$ overlap by~87--93\% across
  different intervals~$I_j$, so summing the per-interval guarantees
  \emph{double-counts} the same targets.
\item The global Cauchy--Schwarz ratio $E_1^2 / (|S| \cdot E_2)$ oscillates
  around~$1.0$ for adversarial subsets~$S$, falling as low as $\sim\!0.988$
  at some values of~$n$---below the threshold needed for Hall's condition.
\item The worst-case K\"onig deficient set spans \emph{all} intervals at
  $\sim\!60\%$ density per interval and \emph{all} degree classes,
  defeating any partition-based strategy.
\end{itemize}

The root cause is that the minimum degree~$\delta \approx 2M/n - 1$ is small
(growing only as $\sqrt{\ln n / \ln\ln n}$), and for vertices~$k$ near~$n/2$,
the degree~$d(k) \approx \delta$ is comparable to the maximum codegree~$D_2$.
This means $D/D_2 \not\to \infty$, which is precisely the regime where
probabilistic methods (LLL, nibble, Janson) and algebraic methods (spectral
gap) lose their power.

\subsection{Paper organization}

\begin{itemize}
\item Section~\ref{sec:construction}: The bipartite graph construction and
  its degree structure.
\item Section~\ref{sec:proved}: What is proved---HK verification, per-interval
  CS, $V_{\min}$ disjointness, FMC theorem.
\item Section~\ref{sec:gap}: The critical gap between per-interval and global
  Hall's condition.
\item Section~\ref{sec:cs-family}: The Cauchy--Schwarz family (6+ variants).
\item Section~\ref{sec:matching}: Matching and fractional methods (8+ variants).
\item Section~\ref{sec:probabilistic}: Probabilistic methods (LLL, Janson,
  nibble).
\item Section~\ref{sec:sieve}: Sieve and inclusion-exclusion methods.
\item Section~\ref{sec:graph-theoretic}: Graph-theoretic and spectral methods.
\item Section~\ref{sec:phenomena}: Interesting computational phenomena.
\item Section~\ref{sec:initial}: Results from the initial proof attempt.
\item Section~\ref{sec:conclusion}: Conclusion---where the snakes lie.
\end{itemize}

All computational experiments are reproducible from scripts included in the
\texttt{experiments/} directory of the accompanying code repository. Every
claim in this paper is traced to a specific verification script or state file.
